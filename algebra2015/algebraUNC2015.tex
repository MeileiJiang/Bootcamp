\documentclass[12pt,oneside]{article}
\usepackage{graphicx, fancyhdr, amssymb, amsmath, geometry, amsfonts, enumerate}
\geometry{verbose,letterpaper,tmargin=1in,bmargin=1in,lmargin=1in,rmargin=1in}
%\renewcommand{\baselinestretch}{2} % Double spaces
\newcommand{\HRule}{\rule{\linewidth}{0.5mm}}


%Defines general exercise environment
\newenvironment{exr}[1]
{\refstepcounter{exrcount}
{\vspace{10pt} \par  \noindent \textbf{\thesection.\theexrcount } \emph{#1}}
}


\DeclareMathOperator{\ac}{ac}
 \DeclareMathOperator{\aff}{aff}
 \newcommand{\argmax}{\operatornamewithlimits{argmax}}
 \DeclareMathOperator{\bd}{bdry}
 \DeclareMathOperator{\bc}{bc}
 \newcommand\bR{\mathbb{R}}
 \DeclareMathOperator{\ca}{\mathcal{A}}
 \DeclareMathOperator{\cconv}{cconv}
  \newcommand{\cF}{\mathcal{F}}
 \DeclareMathOperator{\cl}{cl}
 \DeclareMathOperator{\cN}{\mathcal{N}}
 \DeclareMathOperator{\cone}{cone}
 \DeclareMathOperator{\conv}{conv}
 \DeclareMathOperator{\dom}{dom}
 \newcommand{\ec}{\stackrel{e}{\rightarrow}}
 \DeclareMathOperator{\epi}{epi}
 \newcommand{\fdot}{\,\cdot\,}
 \DeclareMathOperator{\gph}{gph}
 \DeclareMathOperator{\iB}{\Int\mathbb{B}}
 \DeclareMathOperator{\id}{id}
 \DeclareMathOperator{\im}{im}
 \DeclareMathOperator{\Int}{int}
 \newcommand{\ip}[2]{\ensuremath{\langle #1,#2\rangle}}
 \DeclareMathOperator{\lev}{lev}
 \DeclareMathOperator{\lin}{lin}
 \DeclareMathOperator{\lsc}{lsc}
 \DeclareMathOperator{\mB}{\mathbb{B}}
% \newcommand{\mI}{\mathfrak{I}}
 \newcommand{\mU}{\mathcal{U}}
 \DeclareMathOperator{\Par}{par}
 \DeclareMathOperator{\pos}{pos}
 \DeclareMathOperator{\range}{range}
 \newcommand{\Rset}{\mathbb{R}}
 \newcommand{\Rbar}{{\bar\Rset}}
 \DeclareMathOperator{\rc}{rc}
 \DeclareMathOperator{\rb}{rb}
 \DeclareMathOperator{\rec}{rec}
 \DeclareMathOperator{\ri}{ri}
 \DeclareMathOperator{\skel}{skel}
 \DeclareMathOperator{\Span}{span}

\begin{document}


\newcounter{exrcount}[section]
%\renewcommand{\p@exrcount}{\thesection.}

\makeatletter
\newenvironment{Assumptions}
  {\renewcommand\labelenumi{\bfseries (A\theenumi)}%
    \renewcommand\p@enumi{AMon}%
    \begin{enumerate}%
    \addtolength{\itemindent}{2.5em}
  }
  {\end{enumerate}}
\makeatother


%\title{Linear Algebra}


%\date{July 28, 2006}
%\maketitle


\begin{titlepage}

\begin{center}


% Upper part of the page
\includegraphics[width=.5\textwidth]{logo}\\[1cm]

\textsc{\LARGE Department of Biostatistics }\\[.25cm]
\textsc{\LARGE Department of Stat. and OR }\\[1.5cm]

\textsc{\Large Refresher course, Summer 2015}\\[0.5cm]


% Title
\HRule \\[0.4cm]
{ \huge \bfseries Linear Algebra}\\[0.4cm]

\HRule \\[1.5cm]

% Author and supervisor
\begin{minipage}{0.4\textwidth}
\begin{flushleft} \large
\emph{Original Author:} \\
Oleg \textsc{Mayba}\\(UC Berkeley, 2006) 
\end{flushleft}
\end{minipage}
\begin{minipage}{0.5\textwidth}
\begin{flushright} \large
\emph{Modified By:}\\
Eric Lock (UNC, 2010 \& 2011)\\
Gen Li (UNC, 2012)\\
Michael Lamm (UNC, 2013)\\
Wen Jenny Shi (UNC, 2014)\\
Meilei Jiang (UNC, 2015)
\end{flushright}
\end{minipage}\\
\vspace{30 pt}
\begin{centering}
\large \emph{Instructor:} \\
Meilei Jiang \\ (UNC at Chapel Hill)
\end{centering}

\vfill
\textsc{\Large Based on the NSF sponsored (DMS Grant No 0130526) VIGRE Boot camp lecture notes in the Department of Statistics,  University of California, Berkeley }\\[0.5cm]
% Bottom of the page
{\large \today}

\end{center}

\end{titlepage} 

\tableofcontents

\newpage

\section{Introduction}
These notes are intended for use in the warm-up camp for incoming
UNC STOR and Biostatistics graduate students. Welcome to Carolina!  

We assume that you have taken a linear algebra course before and that most of the material in these notes will be a review of what you've already known.  If some of the material is unfamiliar, do not be intimidated!  We hope you find these notes helpful! If not, you can consult the references listed at the end, or any other textbooks of your choice for more information or another style of presentation (most of the proofs on linear algebra part have been adopted from Strang, the proof of F-test from Montgomery et al, and the proof of bivariate normal density from Bickel and Doksum). 

Linear algebra is an important and fundamental math tool for probability, statistics, numerical analysis and operations research. Lots of material in this notes will show up in your future study and research. There will be 9 algebraic classes in total (one class per weekday for two weeks, excluding a day for the university orientation). Each class will last two hours with a short break in between.

Go Tar Heels!

\clearpage
\newpage

\section{Vector Spaces}
A set $V$ is a \textbf{vector space} over $\mathbb{R}$, and its elements
are called \textbf{vectors}, if there are 2 operations defined on it:
\begin{enumerate}
\item Vector addition, that assigns to each pair of vectors $v_{1},
  v_{2} \in V$ another vector $w \in V$ (we write $v_{1} + v_{2} = w$)
\item Scalar multiplication, that assigns to each vector $v \in V$ and
  each scalar $r \in \mathbb{R}$ another vector $w \in V$ (we write $r v
  = w$)
\end{enumerate}
that satisfy the following 8 conditions $\forall$ $ v_{1}, v_{2}, v_{3}
\in V$ and $\forall$ $r_{1}, r_{2} \in \mathbb{R}$:
\begin{enumerate}
\item Commutativity of vector addition: $v_{1} + v_{2}$ $= v_{2} + v_{1}$
\item Associativity of vector addition: $(v_{1} + v_{2}) + v_{3}$ $= v_{1} + (v_{2} + v_{3})$
\item Identity element of vector addition: $\exists$ vector $0 \in V$, s.t. $v + 0 = v$,  $\forall$ $v \in V$
\item Inverse elements of vector addition: $\forall$ $v \in V$ $\exists$ $-v = w \in V$ s.t. $v + w = 0$
\item Compatibility of scalar multiplication with field multiplication: $r_{1}(r_{2} v)$ $= (r_{1} r_{2})v$,  $\forall$ $v \in V$
\item Distributivity of scalar multiplication with respect to field addition: $(r_{1} + r_{2})v$ $= r_{1} v + r_{2} v$,  $\forall$ $v \in V$
\item Distributivity of scalar multiplication with respect to vector addition: $r (v_{1} + v_{2})$ $= r v_{1} + r v_{2}$,  $\forall$ $r \in
  \mathbb{R}$
\item Identity element of scalar multiplication: $1 v$ $= v$,  $\forall$ $v \in V$
\end{enumerate}

Vector spaces over fields other than $\mathbb{R}$ are defined
similarly, with the multiplicative identity of the field replacing 1. We won't concern ourselves with those spaces,
except for when we'll be needing complex numbers later on. Also, we'll
be using the symbol $0$ to designate both the number $0$ and the vector $0$ in
V, and you should always be able to tell the difference from the
context. Sometimes, we'll emphasize that we're dealing with, say, $n
\times 1$ vector $0$ by writing $0_{n \times 1}$.\\

\noindent Examples:
\begin{enumerate}
\item Vector space ${\mathbb{R}}^{n}$ with usual operations of
  element-wise addition and scalar multiplication. An example of these
  operations in $\mathbb{R}^2$ is illustrated above.
\begin{figure}

\[
\setlength{\unitlength}{10pt}
\begin{picture}(10,10)
\put(0,0){\line(1,0){10}}
\put(0,0){\line(0,1){10}}
\put(0,0){\vector(0,1){5}}
\put(0,0){\vector(1,0){5}}
\put(0.3,2){$\bf v$}
\put(2,0.3){${\bf w}$}
\put(0,5){\vector(1,0){5}}
\put(0,0){\vector(1,1){5}}
\put(4,3.5){$\bf v + w$}
\put(0,0){\vector(1,0){10}}
\put(8, 0){$\bf 2w$}
\end{picture}
\]
\caption{Vector Addition and Scalar Multiplication}\label{span1}
\end{figure}

\item Vector space $F_{[-1,1]}$ of all functions defined on interval
  $[-1, 1]$, where we define $(f + g) (x)$ $= f(x) + g(x)$ and $(r f)
  (x)$ $= r f(x)$.
\end{enumerate}


\subsection{Basic Concepts}
We say that $S \subset V$ is a \textbf{subspace} of $V$, if $S$ is
closed under vector addition and scalar multiplication, i.e.
\begin{enumerate}
\item $\forall s_{1}, s_{2} \in S$, $s_{1} + s_{2} \in S$
\item $\forall s \in S$, $\forall r \in \mathbb{R}$, $r s \in S$
\end{enumerate}
You can verify that if those conditions hold, $S$ is a vector space in
its own right (satisfies the 8 conditions above). Note also that $S$
has to be non-empty; the empty set is not allowed as a subspace.\\

\noindent Examples:
\begin{enumerate}
\item A subset $\{0\}$ is always a subspace of a vectors space $V$.
\item Given a set of vectors $ S \subset V$, \textbf{span}\((S)=\{w : w=\sum_{i=1}^n r_iv_i, r_i \in \mathbb{R}, \text{ and } v_i \in S\} \), the set of
  all linear combinations of elements of \(S\) (see below for definition) is a
  subspace of $V$.
\item $S = \{(x,y) \in {\mathbb{R}}^{2}$ $: y = 0 \}$ is a subspace of
  ${\mathbb{R}}^{2}$ (x-axis).
\item A set of all continuous functions defined on interval $[-1, 1]$
  is a subspace of $F_{[-1, 1]}$.

\end{enumerate}
For all of the above examples, you should check for yourself that they
are in fact subspaces. \\

Given vectors $v_{1}, v_{2}, \ldots, v_{n} \in V$, we say that $w \in
V$ is a \textbf{linear combination} of $v_{1}, v_{2}, \ldots, v_{n}$
if for some $r_{1}, r_{2}, \ldots, r_{n} \in \mathbb{R}$, we have $w =
r_{1} v_{1} + r_{2} v_{2} + \ldots + r_{n} v_{n}$. If every vector in
$V$ is a linear combination of $S=\{v_{1}, v_{2}, \ldots, v_{n}\}$,
we have span\((S)=V\), then we say \(S\) \textbf{spans} $V$.\\


\noindent Some properties of subspaces:
\begin{enumerate}
\item Subspaces are closed under linear combinations.
\item A nonempty set $S$ is a subspace if and only if every linear
combination of (finitely many) elements of $S$ also belongs to $S$. \\
  \end{enumerate}

Given vectors $v_{1}, v_{2}, \ldots, v_{n} \in V$ we say that $v_{1},
v_{2}, \ldots, v_{n}$ are \textbf{linearly independent} if $r_{1}
v_{1} + r_{2} v_{2} + \ldots + r_{n} v_{n} = 0$ $\Longrightarrow$
$r_{1} = r_{2} = \ldots = r_{n} = 0$, i.e. the only linear combination
of $v_{1}, v_{2}, \ldots, v_{n}$ that produces $0$ vector is the
trivial one. We say that $v_{1}, v_{2}, \ldots, v_{n}$ are
\textbf{linearly dependent} otherwise.\\

\textbf{Theorem}: Let \(I,S \subset V\) be such that \( I \) is linearly independent, and \( S \) spans \(V\). Then for every \(x\in I \) there exists a \(y \in S \) such that \( \{y\} \cup I \backslash \{x\}\) is linearly independent. \\

\textbf{Proof}: This proof will be by contradiction, and use two facts that can be easily verified from the definitions above. First, if \(I \subset V\) is linearly independent, then \( I \cup \{x\} \) is linearly dependent if and only if (iff) \(x \in \text{span}(I) \). Second, if  \(S,T \subset V\) with \(T \subset \text{span}(S) \) then \(\text{span}(T) \subset \text{span}(S) \).

If the theorem's claim does not hold. Then there exists a \( x \in I \) such that for all \( y \in S \) \( \{y\} \cup I \backslash \{x\}\) is linearly dependent. Let \( I'=I \backslash \{x\} \). By \(I \) linearly independent it follows that \( I' \) is also linearly independent. Then by the first fact above, \(\{y\} \cup I' \) linearly dependent implies \( y \in \text{span}(I') \). Moreover, this holds for all \( y \in S \) so \( S \subset \text{span}(I') \).

By the second fact we then have that \( \text{span}(S) \subset \text{span}(I') \). Now since \( S \) spans \(V\) it follows that \(x \in V =\text{span}(S) \subset \text{span}(I') =\text{span}(I\backslash \{x\}) \). This means there exists \(v_1,v_2,\dots,v_n \in I \backslash \{x\} \) and \(r_1,r_2,\dots,r_n \in \mathbb{R} \) such that \(0 =x- \sum_{i=1}^{n} r_iv_i \), contradicting \(I \) linearly independent. $\Box$\\


\textbf{Corollary}: Let \(I,S \subset V\) be such that \( I \) is linearly independent, and \( S \) spans \(V\). Then \( |I| \le |S| \), where \( |\cdotp|\) denotes the number of elements of a set (possibly infinite).\\

\textbf{Proof}: If \( |S| = \infty \) then the claim holds by convention, and if \( I \subset S \) the claim holds directly. So assume \( |S|=m < \infty \), and \( I \not \subset S \).

Consider now the following algorithm. Select \( x \in I, x\notin S\). By the theorem above, choose a \( y \in S \) such that \( I'= \{y\}\cup I\backslash\{x\} \) is linearly independent. Note that \( |I'|=|I| \) and that \( |I'\cap S| >  |I\cap S| \). If \(I' \subset S \) then the claim holds and stop the algorithm, else continue the algorithm with \(I=I'\).

Now note that the above algorithm must terminate in at most \(m < \infty \) steps. To see this, first note that after the \(m^{\text{\tiny th}} \) iteration \( S \subset I' \). Next, if the algorithm does not terminate at this iteration \( I' \not\subset S \), and there would exist a \(x \in I', x \notin S \). But then since \( S \) spans \(V\) there would exist \(v_1,v_2\dots,v_n \in S \subset I'\) and \(r_1,r_2,\dots,r_n \in \mathbb{R} \) such that \(0=x-\sum_{i=1}^n r_iv_i \) contradicting \(I'\) linearly independent. $\Box$\\


Now suppose that $v_{1}, v_{2}, \ldots, v_{n}$ span $V$ and that,
moreover, they are linearly independent. Then we say that the set $\{
v_{1}, v_{2}, \ldots, v_{n} \}$ is a \textbf{basis} for $V$.\\

\textbf{Theorem}: Let $S$ be a basis for
$V$, and let $T$ be another basis for
$V$. Then $|S| = |T|$.\\

\textbf{Proof}: This follows directly from the above Corollary since \(S\) and \(T\) are both linearly independent, and both span \(V\). $\Box$
\\



We call the unique number of vectors in a basis for $V$ the
\textbf{dimension} of $V$ (denoted dim($V$)).\\

\noindent Examples:
\begin{enumerate}
\item $S =\{0\}$ has dimension 0.
\item Any set of vectors that includes $0$ vector is linearly
  dependent (why?)
\item If $V$ has dimension $n$, and we're given $k < n$ linearly
  independent vectors in $V$, then we can extend this set of vectors
  to a basis.
\item Let $v_1, v_2, \ldots, v_n$ be a basis for $V$. Then if $v \in
  V$, $v = r_1 v_1 + r_2 v_2 + \ldots + r_n v_n$ for some $r_1, r_2,
  \ldots, r_n \in \mathbb{R}$. Moreover, these coefficients are
  unique, because if they weren't, we could also write $v = s_1 v_1 + s_2
  v_2 + \ldots + s_n v_n$, and subtracting both sides we get $0 = v-v
  = (r_1 - s_1) v_1 + (r_2 - s_2) v_2 + \ldots + (r_n - s_n) v_n$, and
  since the $v_i$'s form basis and are therefore linearly independent,
  we have $r_i = s_i$ $\forall i$, and the coefficients are indeed unique.
\item $v_{1} = $ $\left[ \begin{array} {c} 1 \\ 0 \end{array} \right]$ and
  $v_{2} =$ $\left[ \begin{array} {c} -5 \\ 0 \end{array} \right]$ both span
  x-axis, which is the subspace of ${\mathbb{R}}^{2}$. Moreover, any
  one of these two vectors also spans x-axis by itself (thus a
  basis is not unique, though dimension is), and they
  are not linearly independent since $5 v_{1} + 1 v_{2} = 0$
\item $e_{1} =$ $\left[ \begin{array} {c} 1 \\ 0 \\ 0 \end{array}
  \right]$, $e_{2} =$ $\left[ \begin{array} {c} 0 \\ 1 \\ 0
  \end{array} \right]$, and $e_{3} =$ $\left[ \begin{array} {c} 0 \\ 0
  \\ 1 \end{array} \right]$ form the standard basis for ${\mathbb{R}^3}$,
  since every vector $\left[ \begin{array} {c} x_{1} \\ x_{2} \\ x_{3}
  \end{array} \right]$ in ${\mathbb{R}}^{3}$ can be written as $x_{1}
  e_{1} + x_{2} e_{2} + x_{3} e_{3}$, so the three vectors span
  ${\mathbb{R}}^{3}$ and their linear independence is easy to show. In
  general, ${\mathbb{R}}^{n}$ has dimension $n$.
\item Let dim($V$) = $n$, and let $v_{1}, v_{2}, \ldots, v_{m} \in V$,
  s.t. $m > n$. Then $v_{1}, v_{2}, \ldots, v_{m}$ are linearly
  dependent.
\end{enumerate}

\subsection{Special Spaces}
An \textbf{inner product} is a function $f: V \times V \rightarrow \mathbb{R}$
(which we denote by $f(v_{1}, v_{2})$ $= <v_{1}, v_{2}>$),
s.t. $\forall$ $v, w, z \in V$, and $\forall r \in \mathbb{R}$:
\begin{enumerate}
\item $<v, w + r z>$ $= <v, w> + r <v, z>$ (linearity)
\item $<v, w> = <w, v>$ (symmetry)
\item $<v, v> \geq 0$ and $<v, v> = 0$ iff $v = 0$ (positive-definiteness)
\end{enumerate}
We note here that not all vector spaces have inner products defined on
them.
We call the spaces where the inner products are defined the \textbf{inner product space}.\\

\noindent Examples:
\begin{enumerate}
\item Given 2 vectors $x =$ $\left[ \begin{array} {c} x_{1} \\ x_{2} \\
      \vdots \\ x_{n} \end{array} \right]$ and $y =$ $\left[
      \begin{array} {c} y_{1} \\ y_{2} \\ \vdots \\ y_{n} \end{array}
      \right]$ in ${\mathbb{R}}^{n}$, we define their inner product
      $x'y$ $= <x, y>$ $= \displaystyle\sum_{i=1}^{n} {x_{i}
      y_{i}}$. You can check yourself that the 3 properties above are
      satisfied, and the meaning of notation $x' y$ will become clear
      from the next section.
\item Given $f, g \in C_{[-1, 1]}$, we define $<f, g> =$ $\int_{-1}^{1}
  {f(x) g(x) dx}$. Once again, verification that this is indeed an
  inner product is left as an exercise.
\end{enumerate}

\textbf{Cauchy-Schwarz Inequality}: for $v$ and $w$ elements of $V$, the following inequality holds:
\[
<v,w>^2\leq <v,v>\cdot<w,w>
\]
with equality if and only if $v$ and $w$ are linearly dependent.

\textbf{Proof}: Note that $<v,0> = -<v,-0>= -<v,0> \;\Rightarrow\; <v,0>=0,\forall v\in V.$\\
If $w=0$, the equality obviously holds.\\
If $w\neq0$, let $\lambda={<v,w>\over <w,w>}$. Since
\[
0\leq <v-\lambda w,v-\lambda w>=<v,v>-2\lambda<v,w>+\lambda^2<w,w>=<v,v>-{<v,w>^2\over <w,w>}
\]
we can show the result with equality if and only if $v=\lambda w$.
Namely, the inequality holds and it's equality if and only if $v$ and $w$ are linearly dependent.

With Cauchy-Schwarz inequality, we can define the \textbf{angle} between two nonzero vectors $v$ and $w$ as:
\[
angle(v,w)=\arccos{<v,w>\over\sqrt{<v,v>\cdot<w,w>}}
\]
The angle is in $[0,\pi)$.
This generates nice geometry for the inner product space.


The \textbf{norm}, or \textbf{length}, of a vector $v$ in the vector space $V$ is
 a function $g: V \rightarrow \mathbb{R}$ (which we denote by $g(v)= \|v\|$),
s.t. $\forall$ $v, w\in V$, and $\forall r \in \mathbb{R}$:
\begin{enumerate}
\item $\|rv\|$ $= |r|\|v\|$ 
\item $\|v\| \geq 0$, with equality if and only if $v=0$
\item $\|v+ w\| \leq \|v\|+\|w\|$ (triangle inequality)
\end{enumerate}

Again, not all vector spaces have norms defined in them. 
For those with defined norms, they are called the \textbf{normed spaces}.

\noindent Examples:
\begin{enumerate}
\item In ${\mathbb{R}}^n$, let's define the length of a
vector $x$ = $\|x\| = $ $\sqrt{x_{1}^2 + x_{2}^2 + \ldots + x_{n}^2}$ $=
\sqrt{x'x}$, or $\|x\|^2$ $= x'x$. This is called the Euclidian norm, or the $L_2$ norm (denote by $\|x\|_2$). (verify it by yourself)
\item Again in ${\mathbb{R}}^n$, if we define $\|x\|=|x_1|+\ldots+|x_n|$, it's also a norm called the $L_1$ norm (denote by $\|x\|_1$). (verify it by yourself)
\item Given $f\in C_{[-1, 1]}$, we define $\|f\|_p=\left(\int_{-1}^1 |f(x)|^p dx \right)^{1\over p}$, which is also a norm. (see Minkowski Inequality)
\item For any inner product space \( V\), \(\|x\|^2= <x,x> \) defines a norm.
\end{enumerate}

In general, we can naturally obtain a norm from a well defined inner product space.
Let $\|v\|=\sqrt{<v,v>}$ for $\forall v\in V$, where $<\cdot,\cdot>$ is the inner product on the space $V$.
It's not hard to verify all the requirements in the definition of norm (verify it by yourself).
Thus, for any defined inner product, there is a naturally derived norm.
However, in most cases, the opposite (i.e. to obtain inner products from norms) is not obvious.

A more general definition on the vector space is the \textbf{metric}. 
The metric is a function $d: V\times V\rightarrow \mathbb{R}$ such that for $x,y,z \in V$ it satisfies:
\begin{enumerate}
\item $d(x,y)$ $= d(y,x)$
\item $d(x,y) \geq 0$, with equality if and only if $x=y$
\item $d(x,y)\leq d(x,z)+d(y,z)$ (triangle inequality)
\end{enumerate}
Many analytic definitions (e.g. completeness, compactness, continuity, etc) can be defined under metric space.
Please refer to the analysis material for more information.

For any normed space, we can naturally derive a metric as $d(x,y)=\|x-y\|$. This metric is said to be induced by the norm $\|\cdot\|$. However, the opposite is not true. For example, assuming we define the discrete metric on the space $V$, where $d(x,y)=0$ if $x=y$ and $d(x,y)=1$ if $x\neq y$; it is not obvious what kind of norm should be defined in this space.
If a metric $d$ on a vector space $V$ satisfies the properties:
$\forall x, y\in V$ and $\forall r\in\mathbb{R}$,
\begin{enumerate}
\item $d(x,y) = d(x + r, y+r)$ (translation invariance)
\item $d(rx, ry) = |r| d(x, y)$ (homogeneity)
\end{enumerate}
then we can define a norm on $V$ by $\|x\| := d(x,0)$.\\


To sum up, the relation between the three special spaces is as follows.
Given a vector space $V$, if we define an inner product in it, we can naturally derived a norm in it; 
if we have a norm in it, we can naturally derived a metric in it.
The opposite is not true.

\subsection{Orthogonality}
We say that vectors $v, w$ in $V$ are \textbf{orthogonal} if $<v, w> =
0$, or equivalently, $angle(v,w)=\pi/2$. It is denoted as $v\perp w.$\\

\noindent Examples:
\begin{enumerate}
\item In ${\mathbb{R}}^n$ the notion of orthogonality agrees with our
  usual perception of it. If $x$ is orthogonal to $y$, then
  Pythagorean theorem tells us that $\|x\|^2 + \|y\|^2$ $= \|x -
  y\|^2$. Expending this in terms of inner products we get:
\begin{center}
$x'x + y'y =$ $(x-y)'(x-y) =$ $x'x - y'x - x'y + y'y$ or $2 x'y = 0$
\end{center}
and thus $<x, y> = x'y = 0$.
\item Nonzero orthogonal vectors are linearly independent. Suppose we
  have $q_1, q_2, \ldots, q_n$, a set of nonzero mutually orthogonal
  ($<q_i, q_j> = 0$ $\forall i \neq j$) vectors in $V$, and suppose
  that $r_1 q_1 + r_2 q_2 + \ldots + r_n q_n = 0$. Then taking inner
  product of $q_1$ with both sides, we have $r_1 <q_1, q_1> + r_2
  <q_1, q_2> + \ldots + r_n <q_1 q_n> = <q_1, 0> = 0$.  That reduces
  to $r_1 \|q_1\|^2 = 0$ and since $q_1 \neq 0$, we conclude that $r_1 =
  0$. Similarly, $r_i = 0$ $\forall$ $1 \leq i \leq n$, and we
  conclude that $q_1, q_2, \ldots, q_n$ are linearly independent.
\item Suppose we have a $n \times 1$ vector of observations $x =
  \left[ \begin{array} {c} x_1 \\ x_2 \\ \vdots \\ x_n \end{array}
  \right]$. Then if we let $\bar{x} = \frac{1}{n}
  \displaystyle\sum_{i=1}^n {x_i}$, we can see that vector $e = \left[
  \begin{array} {c} x_1 - \bar{x} \\ x_2 - \bar{x} \\ \vdots \\ x_n -
  \bar{x} \end{array} \right]$ is orthogonal to vector $\hat{x} =
  \left[ \begin{array} {c} \bar{x} \\ \bar{x} \\ \vdots \\ \bar{x}
  \end{array} \right]$, since $\displaystyle\sum_{i=1}^n {\bar{x} (x_i
  - \bar{x})} = \bar{x} \displaystyle\sum_{i=1}^n{x_i} - \bar{x}
    \displaystyle\sum_{i=1}^n {\bar{x}} = n {\bar{x}}^2 - n
    {\bar{x}}^2 = 0$.
\end{enumerate}


Suppose $S, T$ are subspaces of $V$. Then we say that they are
\textbf{orthogonal subspaces} if every vector in $S$ is orthogonal to
every vector in $T$. We say that $S$ is the \textbf{orthogonal
  complement} of $T$ in $V$, if $S$ contains ALL vectors orthogonal to
vectors in $T$ and we write $S = T^\perp$. For example, the x-axis and
y-axis are orthogonal subspaces of ${\mathbb{R}}^3$, but they are not
orthogonal complements of each other, since y-axis does not contain
$\left[ \begin{array} {c}  0 \\ 0 \\ 1 \end{array} \right]$, which is
perpendicular to every vector in x-axis. However, y-z plane and x-axis
ARE orthogonal complements of each other in ${\mathbb{R}}^3$. You
should prove as an exercise that if dim$(V)$ $= n$, and dim$(S)$ $=k$,
then dim$(S^\perp)$ $= n-k$.
\subsection{Gram-Schmidt Process}
Suppose we're given linearly independent vectors $v_{1}, v_{2},
\ldots, v_{n}$ in $V$, and there's an inner product defined on $V$. Then
we know that $v_1, v_2, \ldots, v_n$ form a basis for the subspace
which they span (why?).  Then, the \textbf{Gram-Schmidt process} can be used to construct an orthogonal basis for this subspace, as follows:

Let $q_1 = v_1$ Suppose $v_2$ is not orthogonal
to $v_1$. then let $r v_1$ be the \textbf{projection} of $v_2$ on
$v_1$, i.e. we want to find $r \in \mathbb{R}$ s.t. $q_2 = v_2 - r
q_1$ is orthogonal to $q_1$. Well, we should have $<q_1, (v_2 - r
q_1)> = 0$, and we get $r = \frac{<q_1, v_2>}{<q_1, q_1>}$. Notice
that the span of $q_1, q_2$ is the same as the span of $v_1, v_2$,
since all we did was to subtract multiples of original vectors from
other original vectors. Proceeding in similar fashion, we obtain $q_i = v_i -
\left(\left(\frac{<q_1, v_i>}{<q_1, q_1>}\right) q_1 + \ldots +
  \left(\frac{<q_{i-1}, v_{i}>}{<q_{i-1}, q_{i-1}>}\right)
  q_{i-1}\right)$, and we thus end up with an orthogonal basis for the
subspace. If we furthermore divide each of the resulting vectors $q_1,
q_2, \ldots, q_n$ by its length, we are left with \textbf{orthonormal}
basis, i.e. $<q_i, q_j> = 0$ $\forall i \neq j$ and $<q_i, q_i> = 1$
$\forall i$ (why?). We call these vectors that have length 1
\textbf{unit} vectors.

You can now construct an orthonormal basis for the
subspace of $F_{[-1, 1]}$ spanned by $f(x) = 1, g(x) = x$, and $h(x) =
x^2$ (Exercise 2.6 (b)). An important point to take away is that given any basis for
finite-dimensional $V$, if there's an inner product defined on $V$, we
can always turn the given basis into an orthonormal basis.

\subsection*{Exercises}
\addcontentsline{toc}{subsection}{Exercises}

\begin{exr}{}
Show that the space $F_0$ of all differentiable functions $f: \mathbb{R} \rightarrow \mathbb{R}$ with $\frac{df}{dx} = 0$ defines a vector space.
\end{exr}



\begin{exr}{}
Verify for yourself that the two
conditions for a subspace are independent of each other, by coming up with 2 subsets
of ${\mathbb{R}}^{2}$: one that is closed under addition and
subtraction but NOT under scalar multiplication, and one that is
closed under scalar multiplication but NOT under addition/subtraction.
\end{exr}

\begin{exr}{Strang, section 3.5 \#17b}
Let $V$ be the space of all vectors $v = [c_1 \, c_2 \, c_3 \, c_4]' \in {\mathbb{R}}^{4}$ with components adding to 0: $c_1+c_2+c_3+c_4 = 0$.  Find the dimension and give a basis for $V$.
\end{exr}

\begin{exr}{}
Let $v_1,v_2,...,v_n$ be a linearly independent set of vectors in $V$.  Prove that if $n=dim(V)$, $v_1,v_2,...,v_n$ form a basis for $V$.
\end{exr}

\begin{exr}{}
If $F_{[-1,1]}$ is the space of all continuous functions defined on the interval $[-1,1]$,  show that $<f, g> = \int_{-1}^{1} {f(x) g(x) dx}$ defines an inner product of $F_{[-1,1]}$.
\end{exr}

\begin{exr}{}
Parts (a) and (b) concern the space $F_{[-1,1]}$, with inner product $<f,g> = \int_{-1}^1 f(x) g(x) dx$.
\begin{enumerate}[(a)]
	\item Show that $f(x) = 1$ and $g(x) =  x$ are orthogonal in $F_{[-1, 1]}$
	\item Construct an orthonormal basis for the
subspace of $F_{[-1, 1]}$ spanned by $f(x) = 1, g(x) = x$, and $h(x) =
x^2$.
\end{enumerate}
\end{exr}

\begin{exr}{}
If a subspace S is contained in a subspace V, prove that $S^\perp$ contains $V^\perp$.
\end{exr}


\clearpage
\newpage
\section{Matrices and Matrix Algebra}
An $m \times n$ matrix $A$ is a rectangular array of numbers that has $m$
rows and $n$ columns, and we write:
\begin{center}
$A = \left[ \begin{array} {l c c r} a_{11} & a_{12} & \ldots & a_{1n}  \\
    a_{21} & a_{22} & \ldots & a_{2n} \\ \vdots & \ddots & \ddots & \vdots
    \\ a_{m1} & a_{m2} & \ldots & a_{mn} \end{array} \right]$
\end{center}
For the time being we'll restrict ourselves to real matrices, so
$\forall$ $1 \leq i \leq m$ and $\forall$ $1 \leq j \leq n$, $a_{ij}
\in \mathbb{R}$. Notice that a familiar vector $x = \left[
  \begin{array} {c} x_1 \\ x_2 \\ \vdots \\ x_n \end{array} \right]$
$\in {\mathbb{R}}^n$ is just a $n \times 1$ matrix (we say $x$ is a
\textbf{column vector}. A $1 \times n$ matrix is referred to as a
\textbf{row vector}. If $m = n$, we say that $A$ is
\textbf{square}.

\subsection{Matrix Operations}

\noindent \emph{Matrix addition}

Matrix addition is defined elementwise, i.e. $A + B = C$, where
$c_{ij} = a_{ij} + b_{ij}$. Note that this implies that $A + B$ is
defined only if $A$ and $B$ have the same dimensions. Also, note that
$A + B = B + A$. \\

\noindent \emph{Scalar multiplication}

Scalar multiplication is also defined elementwise. If $r \in
\mathbb{R}$, then $r A = B$, where $b_{ij} = r a_{ij}$. Any matrix can
be multiplied by a scalar. Multiplication by 0 results in zero matrix,
and multiplication by 1 leaves matrix unchanged, while multiplying $A$
by -1 results in matrix $-A$, s.t. $A + (-A) = A - A = 0_{m \times
  n}$. You should check at this point that a set of all $m \times n$
matrices is a vector space with operations of addition and scalar
multiplication as defined above.\\

\noindent \emph{Matrix multiplication}

Matrix multiplication is trickier. Given a $m \times n$ matrix $A$
and a $p \times q$ matrix $B$, $AB$ is only defined if $n = p$. In
that case we have $AB = C$, where $c_{ij} =
\displaystyle\sum_{k=1}^{n} {a_{ik} b_{kj}}$, i.e. the $i,j$-th
element of $AB$ is the inner product of the $i$-th row of $A$ and
$j$-th column of $B$, and the resulting product matrix is $m \times
q$. You should at this point come up with your own examples of $A, B$
s.t both $AB$ and $BA$ are defined, but $AB \neq BA$. Thus matrix
multiplication is, in general, non-commutative. Below we list some
very useful ways to think about matrix multiplication:
\begin{enumerate}
\item Suppose $A$ is $m \times n$ matrix, and $x$ is a $n \times 1$
  column vector. Then if we let $a_1, a_2, \ldots, a_n$ denote the respective
  columns of $A$, and $x_1, x_2, \ldots, x_n$ denote the components  of
  $x$,  we get a $m \times 1$ vector $Ax = $ $x_1 a_1 + x_2
  a_2 + \ldots + x_n a_n$, a linear combination of the columns of
  $A$. Thus applying matrix $A$ to a vector always returns a vector
  in the column space of $A$ (see below for definition of column
  space).
\item Now, let $A$ be $m \times n$, and let $x$ be a $1 \times m$ row
  vector. Let $a_1, a_2, \ldots, a_m$ denote rows of $A$, and $x_1,
  x_2, \ldots, x_m$ denote the components of $x$. Then multiplying $A$
  on the left by $x$, we obtain a $1 \times n$ row vector $x A =$ $x_1
  a_1 + x_2 a_2 + \ldots + x_m a_m$, a linear combination of the rows
  of $A$. Thus multiplying matrix on the right by a row vector always
  returns a vector in the row space of $A$ (see below for definition
  of row space)
\item Now let $A$ be $m \times n$, and let $B$ be $n \times k$, and
  let $a_1, a_2, \ldots, a_n$ denote columns of $A$ and  $b_1, b_2,
  \ldots, b_k$ denote the columns of $B$, and let $c_j$ denote the
  $j$-th column of $m \times k$ $C = AB$. Then $c_j = A b_j = b_{1j}
  a_1 + b_{2j} a_2 + \ldots +  b_{nj} a_n$, i.e. we get the columns of
  product matrix by applying $A$ to the columns of $B$. Notice that it
  also implies that every column of product matrix is a linear
  combination of columns of $A$.
\item Once again, consider $m \times n$ $A$ and $n \times k$ $B$, and
  let $a_1, a_2, \ldots a_n$ denote rows of $A$ (they are, of course,
  just $1 \times n$ row vectors). Then letting $c_i$ denote the $i$-th
  row of $C = AB$, we have $c_i = a_i B$, i.e. we get the rows of the
  product matrix by applying rows of $A$ to $B$. Notice, that it means
  that every row of $C$ is a linear combination of rows of $B$.
\item Finally, let $A$ be $m \times n$ and $B$ be $n \times k$. Then
  if we let $a_1, a_2, \ldots, a_n$ denote the columns of $A$ and
  $b_1, b_2, \ldots, b_n$ denote the rows of $B$, then $AB = a_1 b_1 +
  a_2 b_2 + \ldots + a_n b_n$, the sum of $n$ matrices, each of which
  is a product of a row and a column (check this for yourself!).
\end{enumerate}
Let $A$ be $m \times n$, then the \textbf{transpose} of $A$ is
the $n \times m$ matrix $A'$, s.t. $a_{ij} = a_{ji}'$. Now the
notation we used to define the inner product on ${\mathbb{R}}^n$ makes
sense, since given two $n \times 1$ column vectors $x$ and $y$, their
inner product $<x, y>$ is just $x'y$ according to matrix
multiplication.

Let $I_{n \times n}$, denote the $n \times n$ \textbf{identity}
matrix, i.e. the matrix that has 1's down its main diagonal and 0's
everywhere else (in future we might omit the dimensional subscript and
just write $I$, the dimension should always be clear from the
context). You should check that in that case, $I_{n \times n} A = A
I_{n \times n} = A$ for every $n \times n$ $A$. We say that $n \times
n$ $A$, has $n \times n$ \textbf{inverse}, denoted $A^{-1}$, if $A
A^{-1} = A^{-1} A = I_{n \times n}$.  If $A$ has inverse, we say that
$A$ is \textbf{invertible}. Not every matrix has inverse, as you can
easily see by considering the $n \times n$ zero matrix. A square matrix 
that is not invertible is called \textbf{singular} or \textbf{degenerate}. We 
will assume that you are familiar with the use of elimination to calculate
inverses of invertible matrices and will not present this
material. The following are some important results about inverses and
transposes:
\begin{enumerate}
\item $(AB)' = B' A'$

\emph{Proof}: Can be shown directly through entry-by-entry comparison of $(AB)'$ and $B'A'$.

\item If $A$ is invertible and $B$ is invertible, then $AB$ is invertible, and $(AB)^{-1} = B^{-1}A^{-1}$.

\emph{Proof}: Exercise 3.1(a).

\item If $A$ is invertible, then $(A^{-1})' = (A')^{-1}$

\emph{Proof}: Exercise 3.1(b).

\item $A$ is invertible iff  $Ax = 0$
  $\Longrightarrow$ $x = 0$ (we say that $N(A) = \{0\}$, where $N(A)$
  is the nullspace of $A$, to be defined shortly).

\emph{Proof}: Assume $A^{-1}$ exists.  Then,
\begin{eqnarray*}&& Ax = 0 \\ &\rightarrow& A^{-1}(Ax) = A^{-1}0 \\ &\rightarrow& x = 0. \end{eqnarray*}

Now, assume $Ax = 0$ implies $x=0$.  Then the columns ${a_1,...,a_n}$ of $A$ are linearly independent and therefore form a basis for $\mathbb{R}^{n}$ (Exercise 2.4).  So, if \\ $e_1 = \left[ \begin{array} {c} 1 \\ 0 \\ \vdots \\ 0 \end{array} \right]$, $ e_2 = \left[\begin{array} {c} 0 \\ 1 \\ \vdots \\ 0 \end{array} \right]$, ..., $e_n = \left[\begin{array} {c} 0 \\ 0 \\ \vdots \\ 1 \end{array} \right]$, we can write

\[c_{1i}a_1+c_{2i}a_2+...+c_{ni}a_n = A \left[ \begin{array} {c} c_{1,i} \\ c_{2,i} \\ \vdots \\ c_{n,i} \end{array} \right] = e_i \]

for all $i=1,...,n$.  Hence, if $C$ is given by $C_{ij} = c_{ij}$, then

\[AC = [e_1 \, e_2 \, ... \, e_n] = I_n. \]

To see that \(CA=I\), note ACA=IA=A, and therefore A(CA-I)=0. Let \(Z=CA-I\). Then \(Az_i=0\) for each column of \(Z\). By assumption \(Az_i=0 \Rightarrow z_i=0\), so \(Z=0\), \(CA=I\). Hence, $C = A^{-1}$ and \(A\) is invertible $\Box$



\end{enumerate}


\subsection{Special Matrices}

A square matrix $A$ is said to be \textbf{symmetric} if $A = A'$. If
$A$ is symmetric, then $A^{-1}$ is also symmetric (Exercise 3.2). A
square matrix $A$ is said to be \textbf{orthogonal} if $A' =
A^{-1}$. You should prove that columns of an orthogonal matrix are
orthonormal, and so are the rows. Conversely, any square
matrix with orthonormal columns is orthogonal. We note that orthogonal
matrices preserve lengths and inner products: $$<Qx, Qy> = x'Q'Qy =
x'I_{n \times n}y = x'y.$$ In particular $\|Qx\|= \sqrt{x'Q'Qx} =
\|x\|$. Also, if $A$, and $B$ are orthogonal, then so are $A^{-1}$ and
$AB$. We say that a square matrix $A$ is \textbf{idempotent} if $A^2 = A$.

We say that a square matrix $A$ is \textbf{positive definite} if $A$
is symmetric and if $\forall$ $n \times 1$ vectors $x \neq 0_{n \times
  1}$, we have $x'Ax > 0$. We say that $A$ is \textbf{positive
  semi-definite} (or \textbf{non-negative definite} if $A$ is
symmetric and $\forall$ $n \times 1$ vectors $x \neq 0_{n \times 1}$,
we have $x'Ax \geq 0$. You should prove for yourself that every
positive definite matrix is invertible (Exercise 3.3)). One can also show that if $A$ is positive definite, then so is $A'$ (more
generally, if $A$ is positive semi-definite, then so is $A'$).

We say that a square matrix $A$ is \textbf{diagonal} if $a_{ij} = 0$
$\forall$ $i \neq j$. We say that $A$ is \textbf{upper triangular} if
$a_{ij} = 0$ $\forall$ $i > j$. \textbf{Lower triangular} matrices are
defined similarly.\\
We also introduce another concept here: for a square matrix $A$, its
\textbf{trace} is defined to be the sum of the entries on main
diagonal($tr(A) = \displaystyle\sum_{i=1}^{n} {a_{ii}}$). For example,
 $tr(I_{n \times n}) = n$. You may prove for yourself (by method of
 entry-by-entry comparison) that $tr(AB) = tr(BA)$, and $tr(ABC) =
 tr(CAB)$. It's also immediately obvious that $tr(A+B) = tr(A) + tr(B)$.

\subsection{Fundamental Spaces}
Let $A$ be $m \times n$. We will denote by $col(A)$ the subspace of
${\mathbb{R}}^m$ that is spanned by columns of $A$, and we'll call
this subspace the \textbf{column space} of $A$. Similarly, we define the
\textbf{row space} of $A$ to be the subspace of ${\mathbb{R}}^n$
spanned by rows of $A$ and we notice that it is precisely
$col(A')$.

Now, let $N(A) = \{x \in {\mathbb{R}}^n$ $: Ax = 0\}$. You should
check for yourself that this set, which we call \textbf{kernel} or
\textbf{nullspace} of $A$, is indeed subspace of
${\mathbb{R}}^n$. Similary, we define the \textbf{left nullspace} of
$A$ to be $\{x \in {\mathbb{R}}^m$ $: x'A = 0\}$, and we notice that
this is precisely $N(A')$.

The fundamental theorem of linear algebra states:
\begin{enumerate}
\item dim$(col(A))$ = $r$ = dim$(col(A'))$. Dimension of column space
  is the same as dimension of row space. This dimension is called the
  \textbf{rank} of $A$.
\item $col(A) = {(N(A'))}^\perp$ and $N(A) = {(col(A'))}^\perp$. The
  columns space is the orthogonal complement of the left nullspace in
  ${\mathbb{R}}^m$, and the nullspace is the orthogonal complement of
  the row space in ${\mathbb{R}}^n$. We also conclude that dim$(N(A))$
  $= n - r$, and dim$(N(A'))$ $= m - r$.
\end{enumerate}
We will not present the proof of the theorem here, but we hope you are
familiar with these results. If not, you should consider taking a
course in linear algebra (math 383).

We can see from the theorem, that the columns of $A$ are linearly
independent iff the nullspace doesn't contain any vector other than
zero. Similarly, rows are linearly independent iff the left nullspace
doesn't contain any vector other than zero.

We now make some remarks about solving equations of the form $Ax = b$,
where $A$ is a $m \times n$ matrix, $x$ is $n \times 1$ vector, and
$b$ is $m \times 1$ vector, and we are trying to solve for $x$. First
of all, it should be clear at this point that if $b \notin col(A)$,
then the solution doesn't exist. If $b \in col(A)$, but the columns of
$A$ are not linearly independent, then the solution will not be
unique. That's because there will be many ways to combine columns of
$A$ to produce $b$, resulting in many possible $x$'s. Another way to
see this is to notice that if the columns are dependent, the nullspace
contains some non-trivial vector $x^*$, and if $x$ is some solution to
$Ax = b$, then $x + x^*$ is also a solution. Finally we notice that if
$r = m > n$ (i.e. if the rows are linearly independent), then the
columns MUST span the whole ${\mathbb{R}}^n$, and therefore a solution
exists for every $b$ (though it may not be unique).

We conclude then, that if $r = m$, the solution to $Ax = b$ always
exists, and if $r=n$, the solution (if it exists) is unique. This
leads us to conclude that if $n = r = m$ (i.e. $A$ is full-rank square
matrix), the solution always exists and is unique. The proof based on
elimination techniques (which you should be familiar with) then
establishes that a square matrix $A$ is full-rank iff it is
invertible.

We now give the following results:
\begin{enumerate}
\item rank($A'A$) = rank($A$). In particular, if rank($A$) = n
  (columns are linearly independent), then $A'A$ is
  invertible. Similarly, rank($AA'$) = rank($A$), and if the
  rows are linearly independent, $AA'$ is invertible.

\emph{Proof}: Exercise 3.5.

\item $N(AB) \supset N(B)$

\emph{Proof}: Let $x \in N(B)$.  Then, \[(AB)x = A(Bx) = A 0 = 0, \]
so $x \in N(AB)$. $\Box$

\item $col(AB) \subset col(A)$, the column space of product is
  subspace of column space of $A$.

\emph{Proof}: Note that

\[col(AB) = N((AB)')^\perp = N(B'A')^\perp \subset N(A')^\perp = col(A) . \hspace{10 pt} \Box \]

\item $col((AB)') \subset col(B')$, the row space of product is subspace of row space of $B$.

\emph{Proof}: Similar to (3).

\end{enumerate}

\subsection*{Exercises}
\addcontentsline{toc}{subsection}{Exercises}

\begin{exr}{}
Prove the following results:
\begin{enumerate}[(a)]
\item If $A$ is invertible and $B$ is invertible, then $AB$ is
  invertible, and $(AB)^{-1} = B^{-1} A^{-1}$
\item If $A$ is invertible, then $(A^{-1})' = (A')^{-1}$
\end{enumerate}
\end{exr}

\begin{exr}{}
Show that if $A$ is symmetric, then $A^{-1}$ is also symmetric.
\end{exr}

\begin{exr}{}
Show that any positive definite matrix $A$ is invertible (think about nullspaces).
\end{exr}{}

\begin{exr}{Horn \& Johnson 1.2.2}
For $A:n \times n$ and invertible $S: n \times n$, show that $tr(S^{-1}AS) = tr(A)$.  The matrix $S^{-1}AS$ is called a \textbf{similarity} of A.
\end{exr}

\begin{exr}{}
Show that rank($A'A$) = rank($A$). In particular, if rank($A$) = n
  (columns are linearly independent), then $A'A$ is
  invertible. Similarly, show that rank($AA'$) = rank($A$), and if the
  rows are linearly independent, $AA'$ is invertible. (Hint: show that
  the nullspaces of the two matrices are the same).
\end{exr}

\clearpage
\newpage
\section{Projections and Least Squares Estimation}
\subsection{Projections}
In an inner product space, suppose we have $n$ linearly independent vectors $a_1, a_2, \ldots,
a_n$ in ${\mathbb{R}}^m$, and we want to find the projection of a
vector $b$ in ${\mathbb{R}}^m$ onto the space spanned by $a_1, a_2,
\ldots, a_n$, i.e. to find some linear combination $x_1 a_1 + x_2 a_2
+ \ldots + x_n a_n =$ $b^*$, s.t. $<b^*,b - b^*>=0$. It's
clear that if $b$ is already in the span of $a_1, a_2, \ldots, a_n$,
then $b^* = b$ (vector just projects to itself), and if $b$ is
perpendicular to the space spanned by $a_1, a_2, \ldots, a_n$, then
$b^* = 0$ (vector projects to the zero vector).

\textbf{Hilbert Projection Theorem}: Assume $V$ is a Hilbert space (complete inner product space) and $S$ is a closed convex subset of $V$.
For any $v\in V$, there exist an unique $s^*$ in $S$ s.t. 
\[
s^*=\arg\min\limits_{s\in S} \|v-s\|
\]
The vector $s^*$ is called the \textbf{projection} of the vector $v$ onto the subset $S$.

\textbf{Proof (sketch)}: 
 
 First construct a sequence \(y_n \in S\) such that 
\[\| y_n-v\| \to \inf\limits_{s\in S} \|v-s\|.\] 
Then use the Parallelogram Law ( \( \frac{1}{2} \|x-y\|^2 +\frac{1}{2} \|x+y\|^2=\|x\|^2+\|y\|^2 \) ) with \(x=y_n -v\) and \(y=v-y_m\). Rearranging terms, using convexity and appropriate bounds, take the \( \lim\inf \) of each side to show that the sequence \( \{y_n\}_{n=1}^{\infty} \) is Cauchy. This combined with \(V\) complete gives the existence of 
\[
s^*=\arg\min\limits_{s\in S} \|v-s\|.
\]
To obtain uniqueness use the Parallelogram Law and the convexity of \(S\).    \(\Box\)\\

\noindent \textbf{Corollary:} Assume that \(V\) is as above, and that \( S \) is a closed subspace of \(V\). Then \( s^* \) is the projection of \( v \in V\) onto \(S\) iff \(<v-s^*, s>=0  \hspace{3mm}\forall s \in S \). 

\textbf{Proof:} Let \(v\in V\), and assume that \( s^*\) is the projection of \(v\) onto \(S\). The result holds trivialy if \(s=0\) so assume \(s\ne 0\). Since $s^*-ts\in S$, by the Projection Theorem for all \(s\in S\) the function \( f_s(t) = \| v-s^*+ts\|^2 \) has a minimum at \(t=0\). Rewriting this function we see 
\begin{align*}
 f_s(t) & = \| v-s^*+ts\|^2 \\
& =<v-s^*+ts, v-s^*+ts> \\
& = <ts,ts> -2 <v-s^*,ts> + <v-s^*,v-s^*> \\
&=t^2 \|s\|^2 -2t<v-s^*,s> + \|v-s^*\|^2.
\end{align*} 
Since this is a quadratic function of \(t\) with positive quadratic coefficient, the minimum must occur at the vertex, which implies \( <v-s^*,s>=0\). 

For the opposite direction, note first that the function \(f_s(t)\) will still be minimized at \(t=0\) for all \(s \in S\). Then for any \( s' \in S\) take \(s \in S \) such that \( s=s^*-s'\). Then taking \(t=1\) it follows that 
\[
\|v-s^*\|=f_s(0) \le f_s(1)=\|v-s^*+s^*-s'\|=\|v-s'\|.
\]
Thus by \(s^*\) is the projection of \(v\) onto  \(S\). \( \Box \) \\

The following facts follow from the Projection Theorem and its Corollary. 

\noindent\textbf{Fact 1:} The projection onto a closed subspace \(S \) of \(V\), denoted by \(P_S\), is a linear operator.

\textbf{Proof:} Let \(x,y \in V\) and \(a,b \in \mathbb{R} \). Then for any \(s \in S \)
\begin{align*}
<ax+by-aP_Sx-bP_Sy,s> &=  <a(x-P_Sx),s> + <b(y-P_Sy),s> \\
&= a<x-P_Sx,s> + b<y-P_Sy,s> \\
& = a\cdot 0+b \cdot 0=0.
\end{align*}
Thus by the Corollary \( P_S(ax+by)=aP_Sx+bP_Sy\), and \(P_S\) is linear. \(\Box\) \\

\noindent\textbf{Fact 2:} Let \(S\) be a closed subspace of \(V\). Then every \(v \in V\) can be written uniquely as the sum of \(s_1 \in S\) and \( t_1 \in S^{\perp}\).

\textbf{Proof:} That \( V \subset S + S^{\perp} \) follows from the Corollary and taking \(s_1=P_Sv\) and \(t_1=v-P_Sv\) for any \(v \in V\). To see that this is unique assume that \(s_1,s_2 \in S\) and \(t_1,t_2\in S^{\perp} \) are such that 
\[ s_1+t_1=v=s_2+t_2.\]
Then \(s_1-s_2=t_2-t_1\), with \(s_1-s_2 \in S\) and \( t_2-t_1 \in S^{\perp}\), since each is a subspace of \( V\). Therefore \( s_1-s_2,t_2-t_1 \in S \cap S^{\perp}\) which implies 
\[s_1-s_2=t_2-t_1=0 \text{    or   } s_1=s_2 \text{  and  } t_1=t_2.\; \Box\]


\noindent\textbf{Fact 3:} Let \(S \) and \(V\) be as above. Then for any \(x,y \in V \), \( \|x-y\| \ge \|P_Sx -P_Sy\| \).

\textbf{Proof:} First for any \(a,b \in V\),
\begin{align*}
\|a\|^2=\|a-b+b\|^2&=<a-b+b,a-b+b> \\
&= <a-b,a-b+b>+<b,a-b+b> \\
&= <a-b,a-b>+2<a-b,b>+<b,b> \\
&=\|a-b\|^2+\|b\|^2+2<a-b,b>. 
\end{align*}
Taking \(a=x-y\) and \(b=P_Sx-P_Sy\), Fact 1 and the Corollary imply that \( <a-b,b>=0 \) and thus 
\[ \|x-y\|^2 = \|a\|^2=\|a-b\|^2 +\|b\|^2 \ge \|b\|^2=\|P_Sx-P_Sy \|^2.\; \Box \]\\


Now let us focus on the case when \(V=\mathbb{R}^m\) and \(S=\text{span}\{a_1,\dots,a_n\} \) where \(a_1,\dots,a_n\) are linearly independent. 

\noindent\textbf{Fact 4:} Let \( \{a_1, \dots , a_n\}\)  and \( S\) be as above, and \( A= [a_1 \dots a_n ] \). Then the \textbf{projection matrix} \(P_S=P=A(A'A)^{-1}A'\).

\textbf{Proof:} First \(S=\text{span}\{a_1,\dots,a_n\}=col(A)\) and \(b \in \mathbb{R}^m\). Then \(Pb \in S\) implies that there exists a \(x \in \mathbb{R}^n \) such that \(Ax=Pb\). The Corollary to the Projection Theorem states that \( b-Ax \in col(A)^{\perp}\). The Theorem on fundemental spaces tells us that \( col(A)^{\perp}=N(A') \) and thus 
\[ A'(b-Ax)=0 \Rightarrow  A'Ax=A'b\]
The linear independence of \(\{a_1,\dots,a_n\}\) implies that \(rank(A)=n\), which by previous exercise means \(A'A\) is invertible, so \(x= (A'A)^{-1}A'b\) and thus \(Pb=Ax=A(A'A)^{-1}A'b\).  \(\Box\)\\


 We follow up with some
properties of projection matrices:

\begin{enumerate}
\item P is symmetric and idempotent (what should happen to a vector if
  you project it and then project it again?).

\emph{Proof}: Exercise 4.1(a).

\item $I - P$ is the projection onto orthogonal complement of $col(A)$
  (i.e. the left nullspace of $A$)

\emph{Proof}: Exercise 4.1(b).

\item Given any vector $b \in {\mathbb{R}}^m$ and any subspace $S$ of
  ${\mathbb{R}}^m$, $b$ can be written (uniquely) as the sum of its
  projections onto $S$ and $S^\perp$

\emph{Proof}: Assume $dim(S) = q$, so $dim(S^\perp) = m-q$.  Let $A_S = [a_1 \, a_2 \, ... a_q]$ and $A_{S^\perp} = [a_{q+1} \, ... \, a_m]$ be such that $a_1,...,a_q$ form a basis for $S$ and $a_{q+1}, ..., a_m$ form a basis for $S_\perp$.  By 3, if $P_S$ is the projection onto $col(A_S)$ and $P_{S^\perp}$ is the projection onto $col(A_{S^\perp})$, $\forall b \in  {\mathbb{R}}^m$

\[P_S (b) + P_{S^\perp} (b) = P_S (b) + (I - P_{S^\perp})b = b. \]

As columns of $A_S$ and $A_{S^\perp}$ are linearly independent, the vectors $a_1,a_2,...,a_m$ form a basis of $\mathbb{R}^m$.  Hence,

\[b = P_S(b) + P_{S^\perp}(b) = c_1a_1+...+c_qa_q+c_{q+1}a_{q+1}+...+c_ma_m \]
for unique $c_1,...,c_m$. $\Box$

\item $P(I-P) = (I-P)P = 0$ (what should happen to a vector when it's
  first projected to $S$ and then $S^ \perp?$)

\emph{Proof}: Exercise 4.1(c).

\item $col(P) = col(A)$

\emph{Proof}: Exercise 4.1(d).

\item Every symmetric and idempotent matrix $P$ is a projection.

\emph{Proof}: All we need to show is that when we apply $P$ to a vector $b$, the
  remaining part of $b$ is orthogonal to $col(P)$, so $P$ projects
  onto its column space. Well, $P'(b-Pb) = P'(I - P)b = P(I-P)b = (P -
  P^2)b = 0b = 0$. $\Box$

\item Let $a$ be a vector in ${\mathbb{R}}^m$. Then a projection
  matrix onto the line through $a$ is $P = \frac{a a'}{\|a\|^2}$, and if
  $a = q$ is a unit vector, then $P = qq'$.


\item Combining the above result with the fact that we can always come
  up with an orthonormal basis for ${\mathbb{R}}^m$ (Gram-Schmidt) and
  with the fact about splitting vector into projections, we see that
  we can write $b \in {\mathbb{R}}^m$ as $q_1 q_1' b + q_2  q_2' b +
  \ldots + q_m q_m' b$ for some orthonormal basis $\{q_1, q_2, \ldots,
  q_m\}$.

\item If $A$ is a matrix of rank $r$ and $P$ is the projection on $col(A)$, then $tr(P) = r$.

\emph{Proof}: Exercise 4.1(e).

\end{enumerate}

\subsection{Applications to Statistics}
 Suppose we have a linear model, where we model some response as $$y_i
 = x_{i1} \beta_1 + x_{i2} \beta_2 + \ldots + x_{ip} \beta_p +
 \epsilon_i,$$ where $x_{i1}, x_{i2}, \ldots, x_{ip}$ are the values of
 explanatory variables for observation $i$, $\epsilon_i$ is the error
 term for observaion $i$ that has an expected value of 0, and
 $\beta_1, \beta_2, \ldots, \beta_p$ are the coefficients we're
 interested in estimating. Suppose we have $n > p$ observations. Then
 writing the above system in matrix notation we have $Y  = X \beta +
 \epsilon$, where $X$ is the $n \times p$ matrix of explanatory
 variables, $Y$ and $\epsilon$ are $n \times 1$ vectors of observations and
 errors respectively, and $p \times 1$ $\beta$ is what we're
 interested in. We will furthermore assume that the columns of $X$ are
 linearly independent.

 Since we don't actually observe the values of the error terms, we
 can't determine the value of $\beta$ and have to estimate it. One
 estimator of $\beta$ that has some nice properties (which you will
 learn about)  is leaste squares estimator (LSE) $\hat{\beta}$
 that minimizes 
\[\displaystyle\sum_{i=1}^n {(y_i - \tilde{y_i} )^2},\]
 where $\tilde{y_i}=\displaystyle\sum_{i=1}^p \tilde{\beta_j}x_{ij}$. This is equivilient to minimizing \( \| Y-\tilde{Y}\|^2=\|Y-X\tilde{\beta}\|^2\). It follows that the \textbf{fitted values} associated with the lse satisfy 
\[ \hat{Y}=\min\limits_{\tilde{Y}\in col(X)} \|Y-\tilde{Y}\|^2\]
or that \(\hat{Y} \) is the projection of \(Y\) onto \(col(X)\). It follows then from Fact 4 that the the fitted values and LSE are given by 
\[ \hat{Y}=X(X'X)^{-1}X'Y=HY \text{    and    } \hat{\beta}=(X'X)^{-1}X'Y.\]
The matrix $H = X(X'X)^{-1}X'$ is called the \textbf{hat matrix}. It is an orthogonal projection that maps the observed values to the fitted values. The vector of  \textbf{residuals} \(e = Y - \hat{Y} = (I - H) Y\) are orthogonal to \(col(X)\) by the Corollary to the Projection Theorem, and in particular \(e \perp \hat{Y}\). 


Finally, suppose there's a column $x_j$ in $X$ that is perpendicular to all
 other columns. Then because of the results on the separation of
 projections ($x_j$ is the orthogonal complement in $col(X)$ of the
 space spanned by the rest of the columns), we can project $b$ onto
 the line spanned by $x_j$, then project $b$ onto the space spanned by
 rest of the columns of $X$ and add the two projections together to
 get the overall projected value. What that means is that if we throw
 away the column $x_j$, the values of the coefficients in $\beta$
 corresponding to other columns will not change. Thus inserting or
 deleting from $X$ columns orthogonal to the rest of the column space
 has no effect on estimated coefficients in $\beta$ corresponding to
 the rest of the columns.
 
 Recall that the Projection Theorem and its Corollary are stated in the general setting of Hilbert spaces. One application of these results which uses this generality and arrises in STOR 635 and possibly 654 is the interpretation of conditional expectations as projections. Since this application requires a good deal of material covered in the first semester courses, i.e measure theory and integration, an example of this type will not be given. Instead an example on a simpler class of functions will be given.

\textbf{Example:} Let \(V=C_{[-1,1]}\) with \(\|f\|^2=<f,f>=\int_{-1}^1 f(x)f(x) dx\). Let \(h(x)=1\), \(g(x)=x\) and \(S=\text{span}\{h,g\}=\{\text{all linear functions}\}\). What we will be interested is calculating \(P_Sf\) where \(f(x)=x^2\).

From the Corollary we know that \(P_Sf\) is the unique linear function that satisfies \( <f-P_Sf,s>=0\) for all linear functions \(s\in S\). By previous (in class ) exercise finding \(P_Sf\) requires finding constans \(a\) and \(b\) such that
\[<x^2-(ax+b),1>=0=<x^2-(ax+b),x>\].
First we solve
\begin{align*}
0=<x^2-(ax+b),1>&=\int_{-1}^{1} (x^2-ax-b)\cdot 1 \hspace{1mm} dx\\
&=\left. \left(\frac{x^3}{3}-\frac{ax^2}{2}-bx\right)\right|_{-1}^{1}\\
&=\left(\frac{1}{3}-\frac{a}{2}-b\right)-\left(\frac{-1}{3}-\frac{a}{2}+b\right)\\
&=\frac{2}{3}-2b \Rightarrow b=\frac{1}{3}.
\end{align*}
Next, 
\begin{align*}
0=<x^2-(ax+b),x>&=\int_{-1}^{1} (x^2-ax-b)\cdot x\hspace{1mm} dx\\
&=\int_{-1}^{1}x^3-ax^2-bx \hspace{1mm}dx\\
&=\frac{x^4}{4}-\frac{ax^3}{3}-\frac{bx^2}{2}|_{-1}^{1}\\
&=\left(\frac{1}{4}-\frac{a}{3}-\frac{b}{2}\right)-\left(\frac{1}{4}+\frac{a}{3}-\frac{b}{2}\right)\\
&=\frac{-2a}{3} \Rightarrow a=0.
\end{align*}
Therefore \(P_Sf=ax+b=\frac{1}{3}\)  \(\Box\)\\
\subsection*{Exercises}
\addcontentsline{toc}{subsection}{Exercises}



\begin{exr}{}
Prove the following properties of projection matrices:
\begin{enumerate}[(a)]
\item P is symmetric and idempotent.

\item $I - P$ is the projection onto orthogonal complement of $col(A)$
  (i.e. the left nullspace of $A$)

\item $P(I-P) = (I-P)P = 0$

\item $col(P) = col(A)$

\item If $A$ is a matrix of rank $r$ and $P$ is the projection on $col(A)$, $tr(P) = r$.

\end{enumerate}
\end{exr}

\clearpage
\newpage
\section{Differentiation}
\subsection{Basics}
Here we just list the results on taking derivatives of expressions
with respect to a vector of variables (as opposed to a single
variable). We start out by defining what that actually means: Let $x =
\left[ \begin{array} {c} x_1 \\ x_2 \\ \vdots \\ x_k \end{array}
\right]$ be a vector of variables, and let $f$ be some real-valued
function of $x$ (for example $f(x) = sin(x_2) + x_4$ or $f(x) =
{x_1}^{x_7} + x_{11} log(x_3)$). Then we define $\frac{\partial
  f}{\partial x} = \left[ \begin{array} {c} \frac{\partial f}{\partial
      x_1} \\ \frac{\partial f}{\partial x_2} \\ \vdots \\
    \frac{\partial f}{\partial x_k} \end{array} \right]$. Below are
the extensions

\begin{enumerate}
\item Let $a \in \mathbb{R}^k$, and let $y = a'x = a_1 x_1 + a_2 x_2 +
  \ldots + a_k x_k$. Then $\frac{\partial y}{\partial x} = a$

\emph{Proof}: Follows immediately from definition.

\item Let $y = x'x$, then $\frac{\partial y}{\partial x} = 2x$

\emph{Proof}: Exercise 5.1(a).

\item Let $A$ be $k \times k$, and $a$ be $k \times 1$, and $y =
  a'Ax$. Then $\frac{\partial y}{\partial x} = A'a$

\emph{Proof}:  Note that $a'A$ is $1 \times k$.  Writing $y = a'Ax = (A'a)'x$ it's then clear from 1 that $\frac{\partial y}{\partial x} = A'a$. $\Box$

\item Let $y = x'Ax$, then $\frac{\partial y}{\partial x} = Ax + A'x$
  and if $A$ is symmetric $\frac{\partial y}{\partial x} = 2Ax$. We
  call the expression $x'Ax = \displaystyle\sum_{i=1}^k
  {\displaystyle\sum_{j=1}^k {a_{ij} x_i x_j}}$, a \textbf{quadratic
  form} with corresponding matrix $A$.

\emph{Proof}: Exercise 5.1(b).

\end{enumerate}
\subsection{Jacobian and Chain Rule} \label{Sec5.2}

\noindent
A function $f: \Rset^n \to \Rset^m$ is said to be \textbf{differentiable at $x$} if there exists a linear function
$L: \Rset^n \to \Rset^m$ such that
\[
\lim_{x' \to x, x'\ne x}\frac{f(x')-f(x)-L(x'-x)}{\|x'-x\|} =0.
\]
It is not hard to see that such a linear function $L$, if any, is uniquely defined by the above equation. It is called the differential of $f$ at $x$. Moreover, if $f$ is differentiable at $x$, then all of its partial derivatives exist, and we write the Jacobian matrix of $f$ at $x$ by arranging its partial derivatives into a $m\times n$ matrix,
\[
Df(x) = \begin{bmatrix}
\frac{\partial f_1}{\partial x_1} (x) & \cdots & \frac{\partial f_1}{\partial x_n} (x) \\
\vdots & & \vdots \\
\frac{\partial f_m}{\partial x_1} (x) & \cdots & \frac{\partial f_m}{\partial x_n} (x)
\end{bmatrix}.
\]
It is not hard to see that the differential $L$ is exactly  represented by the Jacobian matrix $Df(x)$. Hence, \[
\lim_{x' \to x, x'\ne x}\frac{f(x')-f(x)-Df(x)(x'-x)}{\|x'-x\|} =0
\]
whenever $f$ is differentiable at $x$.

In particular, if $f$ is of the form $f(x)=Mx + b$, then $Df (x) \equiv M$.

\vspace{4mm}

   Now consider the case where $f$ is a function from $\Rset^n$ to $\Rset$. The Jacobian matrix $Df(x)$ is a $n$-dimensional row vector, whose transpose is the gradient. That is, $Df(x) = \nabla f(x)^T$.
 Moreover, if $f$ is twice differentiable and we define $g (x)= \nabla f(x)$, then Jacobian matrix of $g$ is the Hessian matrix of $f$. That is,
 \[Dg(x) = \nabla^2 f(x).\]

\vspace{4mm}

Suppose that $f:\Rset^n\to \Rset^m$ and $h:\Rset^k \to \Rset^n$ are two differentiable functions. The \emph{chain rule of differentiability} says that the function $g$ defined by $g(x) =f( h(x))$ is also differentiable, with
\[
D g(x) = Df(h(x)) Dh(x).
\]
For the case $k=m=1$, where $h$ is from $\Rset$ to $\Rset^n$ and $f$ is from $\Rset^n$ to $\Rset$, the equation above becomes
\[
g'(x)= Df(h(x)) D h(x) = \ip{\nabla f(h(x))}{Dh(x)} = \sum_{i=1}^n \partial_i f (h(x)) h'_i(x)
\]
where $\partial_i f (h(x))$ is the $i$th partial derivative of $f$ at $h(x)$ and $h'_i(x)$ is the derivative of the $i$th component of $h$ at $x$.

\vspace{4mm}

Finally, suppose that $f:\Rset^n \to \Rset^m$ and $h:\Rset^n \to \Rset^m$ are two differentiable functions, then the function $g$ defined by $g(x)=\ip{f(x)}{h(x)}$ is also differentiable, with
\[
D g(x)= f(x)^T Dh(x)  + h(x)^T Df(x).
\]
Taking transposes on both sides, we get
\[
\nabla g(x)= Dh(x)^T f(x)   + Df(x)^T h(x).
\]

\vspace{4mm}

\noindent
\textbf{Example 1}. Let $f: \Rset^n \to \Rset$ be a differentiable function. Let $x^* \in \Rset^n$ and $d \in \Rset^n$ be fixed.
    Define a function $g:\Rset \to \Rset$ by $g(t) = f(x^* + t d)$. If we write $h(t)=x^*+td$, then $g(t)=f(h(t))$. We have
    \[
    g'(t)= \ip{\nabla f(x^*+td)}{ Dh(t)}=\ip{\nabla f(x^*+td)}{d}.
    \]
    In particular,
    \[
    g'(0)= \ip{\nabla f(x^*)}{d}.
    \]

%    Or
%    \[
%    \begin{split}
%    g'(0) &= \sum_{i=1}^n \frac{\partial f}{\partial x_i} (h(0)) h'_i(0)\\
%    &= \sum_{i=1}^n \frac{\partial f}{\partial x_i} (x^*) d_i \\
%    &=\ip{d}{\nabla f(x^*)}
%    \end{split}
%    \]


   Suppose in addition that $f$ is twice differentiable.
   Write $F(x)=\nabla f(x)$. Then $g'(t)= \ip{d}{F(x^*+td)}= \ip{d}{F(h(t))}=d^T F(h(t))$. We have
    \[
    g''(t)= d^T DF(h(t)) D h(t) =d^T \nabla^2 f (h(t)) d = \ip{d}{\nabla^2 f(x^*+td) d }.
    \]
    In particular,
    \[
    g''(0)= \ip{d}{\nabla^2 f(x^*) d }.
    \]

\vspace{4mm}
\noindent
\textbf{Example 2}. Let $M$ be an $n\times n$ matrix and let $b\in \Rset^n$, and define a function $f: \Rset^n\to \Rset$ by $f(x)= x^T M  x+ b^T x$. Because $f(x) = \ip{x}{Mx+b}$, we have
\[
\nabla f(x)= M^T x + M x + b = (M^T+M) x + b,
\]
and
\[
\nabla^2 f(x)= M^T + M.
\]
In particular, if $M$ is symmetric then $\nabla f(x)=2Mx + b$ and $\nabla^2 f(x) = 2M$.

\subsection*{Exercises}
\addcontentsline{toc}{subsection}{Exercises}

\begin{exr}{}
Prove the following properties of vector derivatives:
\begin{enumerate}[(a)]
\item Let $y = x'x$, then $\frac{\partial y}{\partial x} = 2x$

\item Let $y = x'Ax$, then $\frac{\partial y}{\partial x} = Ax + A'x$
  and if $A$ is symmetric $\frac{\partial y}{\partial x} = 2Ax$.
\end{enumerate}
\end{exr}

\begin{exr}{}
The \textbf{inverse function theorem} states that for a function $f: \mathbb{R}^n \rightarrow \mathbb{R}^n$, the inverse of the Jacobian matrix for $f$ is the Jacobian of $f^{-1}$:

\[(Df)^{-1} = D(f^{-1}). \]

\noindent Now consider the function $f:\mathbb{R}^2 \rightarrow \mathbb{R}^2$ that maps from polar $(r, \theta)$ to cartesian coordinates $(x,y)$:

\[f(r,\theta) = \left[ \begin{array} {c} r \, cos(\theta) \\ r \, sin(\theta) \end{array} \right] = \left[ \begin{array} {c} x \\ y \end{array} \right]. \]

\noindent Find $Df$, then invert the two-by-two matrix to find $\frac{\partial r}{\partial x}$, $\frac{\partial r}{\partial y}$, $\frac{\partial \theta}{\partial x}$, and $\frac{\partial \theta}{\partial y}$.

\end{exr}

\clearpage
\newpage
\section{Matrix Decompositions}
We will assume that you are familiar with $LU$ and $QR$ matrix
decompositions. If you are not, you should look them up, they are easy
to master. We will in this section restrict ourselves to
eigenvalue-preserving decompositions.
\subsection{Determinants}
We will assume that you are familiar with the idea of determinants,
and specifically calculating determinants by the method of cofactor
expansion along a row or a column of a square matrix. Below we list
the properties of determinants of real square matrices. The first 3
properties are defining, and the rest are established from those 3.
\begin{enumerate}
\item det$(A)$ depends linearly on the first row. \\ det$\left[
    \begin{array} {l c c r} a_{11} + a_{11}' & a_{12} + a_{12}' &
      \ldots & a_{1n} + a_{1n}' \\ a_{21} & a_{22} & \ldots & a_{2n} \\
      \vdots & \vdots & \ddots & \vdots \\ a_{n1} & a_{n2} & \ldots &
      a_{nn} \end{array} \right]$ = \\

det$\left[ \begin{array} {l c c r}
      a_{11} & a_{12} & \ldots & a_{1n} \\ a_{21} & a_{22} & \ldots &
      a_{2n} \\ \vdots & \vdots & \ddots & \vdots \\ a_{n1} & a_{n2} &
      \ldots & a_{nn} \end{array} \right]$ + det$\left[ \begin{array} {l c c r}
      a_{11}' & a_{12}' & \ldots & a_{1n}' \\ a_{21} & a_{22} & \ldots &
      a_{2n} \\ \vdots & \vdots & \ddots & \vdots \\ a_{n1} & a_{n2} &
      \ldots & a_{nn} \end{array} \right]$.\\

det$\left[ \begin{array} {l c c r}
      r a_{11} & r a_{12} & \ldots & r a_{1n} \\ a_{21} & a_{22} & \ldots &
      a_{2n} \\ \vdots & \vdots & \ddots & \vdots \\ a_{n1} & a_{n2} &
      \ldots & a_{nn} \end{array} \right]$ = $r$ det$\left[
      \begin{array} {l c c r}  a_{11} & a_{12} & \ldots & a_{1n} \\
      a_{21} & a_{22} & \ldots & a_{2n} \\ \vdots & \vdots & \ddots &
      \vdots \\ a_{n1} & a_{n2} & \ldots & a_{nn} \end{array} \right]$
\item Determinant changes sign when two rows are exchanged. This also
  implies that the determinant depends linearly on EVERY row, since we
  can exhange row$i$ with row 1, split the determinant, and exchange
  the rows back, restoring the original sign.
\item det$(I)$ = 1
\item If two rows of $A$ are equal, det$(A)$ = 0 (why?)
\item Subtracting a multiple of one row from another leaves
  determinant unchanged.

  \emph{Proof}: Suppose instead of row $i$ we now have
  row $i - r j$. Then splitting the determinant of the new matrix
  along this row we have det(original) + det(original matrix with row
  $rj$ in place of row $i$. That last determinant is just $r$ times
  determinant of original matrxi with row $j$ in place of row $i$, and
  since the matrix has two equal rows, the determinant is 0. So the determinant of the new matrix has to be equal to the determinant of the original. $\Box$

\item If a matrix has a zero row, its determinant is 0. (why?)
\item If a matrix is triangular, its determinant is the product of
  entries on main diagonal

\emph{Proof:} Exercise 6.1.

\item det$(A)$ = 0 iff $A$ is not invertible (proof involves ideas of
  elimination)
\item det$(AB) = $ det$(A)$det$(B)$. In particular det$(A^{-1})$ =
  $\frac{1}{det(A)}$.

\emph{Proof}: Suppose det$(B)$ = 0. Then
  $B$ is not invertible, and $AB$ is not invertible (recall $(AB)^{-1} = B^{-1}A^{-1}$, therefore det$(AB)$ = 0. If det$(B) \neq 0$, let $d(A) =
  \frac{det(AB)}{det(B)}$. Then,
\begin{enumerate}[(1)]
	\item For $A_*=[a^*_{11} \, a^*_{12} \, ... \, a^*_{1n}] \in \mathbb{R}^n$ let $A_i$ be the \(i^{\text{\tiny{th}}}\) row of A,  \(r \in \mathbb{R}\), and \(A^*\)  be the matrix \(A\) but with its first row replaced with \(A_*\).  Then,


\begin{align*}
d \left(   \left[ \begin{array}{c} rA_1+A_* \\ \vdots \\ A_n \end{array} \right] \right) &= det\left(\left[ \begin{array}{c} rA_1+A_*  \\ \vdots \\ A_n \end{array} \right] B \right) (det(B))^{-1} \\
& = \frac{\det\left(\left[ \begin{array}{c} (rA_1+A_*)B  \\ \vdots \\ A_nB \end{array} \right] \right)}{det(B)} \\
&=\frac{det\left( \left[ \begin{array}{c} rA_1B  \\ \vdots \\ A_nB \end{array} \right]\right)  +  det \left(\left[ \begin{array}{c}A_*B  \\ \vdots \\ A_nB \end{array} \right] \right) }{det(B)} \\
&= \frac{ r \cdot det \left( AB\right)+ det\left( A^*B\right) }{det(B)}\\
&=r\cdot d(A)+d(A^*).
\end{align*}

Using the same argument for rows \(2,3,\dots,n\) we see that \(d(\cdot)\) is linear for each row.

\item Let \(A^{i,j}\) be the matrix \(A\)  with rows \(i\) and \(j\) interchanged, and WLOG assume \(i < j\). Then

$$
d(A^{i,j})
=\frac{det \left( \left[  \begin{array}{c} A_1 \\ \vdots \\ A_j \\ \vdots \\ A_i \\ \vdots \\ A_n \end{array} \right]B \right) }{det(B)}  
=\frac{ det \left( \left[  \begin{array}{c} A_1B \\ \vdots \\ A_jB \\ \vdots \\ A_iB \\ \vdots \\ A_nB \end{array} \right] \right)  }{det(B)}  
=\frac{det\left( (AB)^{i,j}\right)}{det(B)}=\frac{-det(AB)}{det(b)}=-d(A).
$$




   \item $d(I) = det(IB)/det(B) = det(B)/det(B) = 1$.
\end{enumerate}
So conditions 1-3 are satisfied and therefore $d(A) = det(A)$. $\Box$



\item det$(A')$ = det$(A)$. This is true since expanding along the row
  of $A'$ is the same as expanding along the corresponding column of $A$.
\end{enumerate}
\subsection{Eigenvalues and Eigenvectors}
Given a square $n \times n$ matrix $A$, we say that $\lambda$ is an
\textbf{eigenvalue} of $A$, if for some non-zero $x \in {\mathbb{R}}^n$
we have $Ax = \lambda x$. We then say that $x$ is an
\textbf{eigenvector} of $A$, with corresponding eigenvalue
$\lambda$. For small $n$, we find eigenvalues by noticing that $$Ax =
\lambda x \Longleftrightarrow (A - \lambda I)x = 0
\Longleftrightarrow A - \lambda I$$ is not invertible
$\Longleftrightarrow$ det$(A - \lambda I) = 0$. We then write out the
formula for the determinant (which will be a polynomial of degree $n$
in $\lambda$) and solve it. Every $n \times n$ $A$ then has $n$
eigenvalues (possibly repeated and/or complex), since every polynomial
of degree $n$ has n roots. Eigenvectors for a specific value of
$\lambda$ are found by calculating the basis for nullspace of $A -
\lambda I$ via standard elimination techniques. If $n \geq 5$, there's
a theorem in algebra that states that no formulaic expression for the
roots of the polynomial of degree $n$ exists, so other techniques are
used, which we will not be covering. Also, you should be able to see
that the eigenvalues of $A$ and $A'$ are the same (why? Do the
eigenvectors have to be the same?), and that if $x$ is an eigenvector
of $A$ ($Ax = \lambda x$), then so is every multiple $r x$ of $x$,
with same eigenvalue $(Arx = \lambda r x)$. In particular, a unit
vector in the direction of $x$ is an eigenvector.\\

\textbf{Theorem}: Eigenvectors corresponding to distinct eigenvalues
are linearly independent.\\

\textbf{Proof}: Suppose that there are only two distinct
eigenvalues ($A$ could be $2 \times 2$ or it could have repeated
eigenvalues), and let $r_1 x_1 + r_2 x_2 = 0$. Applying $A$ to both
sides we have $r_1 Ax_1 + r_2 Ax_2 = A 0 = 0$ $\Longrightarrow$
$\lambda_1 r_1 x_1 + \lambda_2 r_2 x_2 = 0$. Multiplying first
equation by $\lambda_1$ and subtracting it from the second, we get
$\lambda_1 r_1 x_1 + \lambda_2 r_2 x_2 - (\lambda_1 r_1 x_1 + \lambda_1
r_2 x_2) = 0 - 0 = 0$ $\Longrightarrow$ $r_2 (\lambda_2 - \lambda_1)
x_2 = 0$ and since $x_1 \neq 0$, and $\lambda_1 \neq \lambda_2$, we
conclude that $r_2 = 0$. Similarly, $r_1 = 0$ as well, and we conclude
that $x_1$ and $x_2$ are in fact linearly independent. The proof
extends to more than 2 eigenvalues by induction. $\Box$
\\

We say that $n \times n$ $A$ is \textbf{diagonalizable} if it has $n$
linearly independent eigenvectors. Certainly, every matrix that has
$n$ DISTINCT eigenvalues is diagonalizable (by the proof above), but
some matrices that fail to have $n$ distinct eigenvalues may still be
diagonalizable, as we'll see in a moment. The reasoning behind the
term is as follows: Let $s_1, s_2, \ldots, s_n \in {\mathbb{R}}^n$ be
the set of linearly independent eigenvectors of $A$, let $\lambda_1,
\lambda_2, \ldots, \lambda_n$ be corresponding eigenvalues (note that
they need not be distinct), and let $S$ be $n \times n$ matrix the $j$-th
column of which is $s_j$. Then if we let $\Lambda$ be $n \times n$
diagonal matrix s.t. the $ii$-th entry on the main diagonal is
$\lambda_i$, then from familiar rules of matrix multiplication we
can see that $AS = S \Lambda$, and since $S$ is invertible (why?) we
have $S^{-1} A S = \Lambda$ (Exercise 6.2). Now suppose that we have $n \times n$ $A$
and for some $S$, we have $S^{-1} A S = \Lambda$, a diagonal
matrix. Then you can easily see for yourself that the columns of $S$
are eigenvectors of $A$ and diagonal entries of $\Lambda$ are
corresponding eigenvalues. So the matrices that can be made into a
diagonal matrix by pre-multiplying by $S^{-1}$ and post-multiplying by
$S$ for some invertible $S$ are precisely those that have $n$ linearly
independent eigenvectors (which are, of course, the columns of
$S$). Clearly, $I$ is diagonalizable ($S^{-1} I S = I$) $\forall$
invertible $S$, but $I$ only has a single eigenvalue 1. So we have an
example of a matrix that has a repeated eigenvalue but nonetheless has
$n$ independent eigenvectors.

If $A$ is diagonalizable, calculation of powers of $A$ becomes very
easy, since we can see that $A^k = S \Lambda^k S^{-1}$, and taking
powers of a diagonal matrix is about as easy as it can get. This is often
a very helpful identity when solving recurrent relationships.
\\

\textbf{Example} A classical example is the Fibonacci sequence 1, 1, 2, 3, 5, 8, \ldots,
where each term (starting with 3rd one) is the sum of the preceding
two: $F_{n+2} = F_{n} + F_{n+1}$. We want to find an explicit formula
for $n$-th Fibonacci number, so we start by writing\\
\begin{center}
$\left[ \begin{array} {c} F_{n+1} \\ F_{n} \end{array} \right] =
\left[ \begin{array} {l r} 1 & 1 \\ 1 & 0 \end{array} \right] \left[
  \begin{array} {c} F_{n} \\ F_{n-1} \end{array} \right]$
\end{center}
or $u_{n} = A u_{n-1}$, which becomes $u_{n} = A^n u_0$, where $A =
\left[ \begin{array} {l r} 1 & 1 \\ 1 & 0 \end{array} \right]$, and
$u_0 = \left[ \begin{array} {c} 1 \\ 0 \end{array}
\right]$. Diagonalizing $A$ we find that $S = \left[ \begin{array} {l
      r} \frac{1 + \sqrt{5}}{2} & \frac{1 - \sqrt{5}}{2} \\ 1 & 1
  \end{array} \right]$ and $\Lambda = \left[ \begin{array} {l r}
    \frac{1 + \sqrt{5}}{2} & 0 \\ 0 & \frac{1-\sqrt{5}}{2} \end{array}
    \right]$, and identifying $F_{n}$ with the second component of
    $u_n = A^n u_0 = S \Lambda^n S^{-1} u_0$, we obtain $F_n =
    \frac{1}{\sqrt{5}} \left[ {\left( \frac{1 + \sqrt{5}}{2} \right)}^n
    - {\left( \frac{1 - \sqrt{5}}{2} \right)}^n \right]$\\
We finally note that there's no relationship between being
    diagonalizable and being invertible. $\left[ \begin{array} {l r} 1
    & 0 \\ 0 & 1 \end{array} \right]$ is both invertible and
    diagonalizable, $\left[ \begin{array} {l r} 0
    & 0 \\ 0 & 0 \end{array} \right]$ is diagonalizable (it's already
    diagonal) but not invertible, $\left[ \begin{array} {l r} 3
    & 1 \\ 0 & 3 \end{array} \right]$ is invertible but not
    diagonalizable (check this!), and $\left[ \begin{array} {l r} 0
    & 1 \\ 0 & 0 \end{array} \right]$ is neither invertible nor
    diagonalizable (check this too).

\subsection{Complex Matrices and Basic Results}
We now allow complex entries in vectors and matrices. Scalar
multiplicaiton now also allows multiplication by complex numbers, so
we're going to be dealing with vectors in ${\mathbb{C}}^n$, and you
should check for yourself that dim$({\mathbb{C}}^n)$ =
dim$({\mathbb{R}}^n)$ = $n$ (Is $\mathbb{R}^n$ a subspace of
$\mathbb{C}^n$?) We also note that we need to tweak a bit the earlier
definition of transpose to account for the fact that if $x = \left[
  \begin{array} {c} 1 \\ i \end{array} \right] \in {\mathbb{C}}^2$,
then $$x'x = 1 + i^2 = 0 \neq 1 = \|x\|^2.$$ We note that in the complex
case $\|x\|^2 = (\bar{x})'x$, where $\bar{x}$ is the complex conjugate
of $x$, and we introduce the notation $x^H$ to denote the
transpose-conjugate $\bar{x}'$ (thus we have $x^H x = \|x\|^2$). You
can easily see for yourself that if $x \in \mathbb{R}^n$, then $x^H =
x'$. $A^H = (\bar{A})'$ for $n \times n$ matrix $A$ is defined
similarly and we call $A^H$ \textbf{Hermitian transpose} of $A$. You
should check that $(A^H)^H = A$ and  that $(AB)^H = B^H A^H$ (you might want to
use the fact that for complex numbers $x, y \in \mathbb{C}$, $\overline{x +
  y} = \bar{x} + \bar{y}$ and $\overline{x y} = \bar{x} \bar{y}$).  We say
that $x$ and $y$ in $\mathbb{C}^n$ are orthogonal if $x^H y = 0$ (note
that this implies that $y^H x = 0$, although it is NOT true in general
that $x^H y = y^H x$).

We say that $n \times n$ matrix $A$ is \textbf{Hermitian} if $A =
A^H$. We say that $n \times n$ $A$ is \textbf{unitary} if $A^H A = A
A^H = I (A^H = A^{-1})$. You should check for yourself that every
symmetric real matrix is  Hermitian, and every orthogonal real matrix
is unitary. We say that a square matrix $A$ is \textbf{normal} if it
commutes with its Hermitian transpose: $A^H A = A A^H$. You should
check for yourself that Hermitian (and therefore symmetric) and
unitary (and therefore orthogonal) matrices are normal.  We next
present some very important results about Hermitian and unitary
matrices (which also include as special cases symmetric and orthogonal
matrices respectively):
\begin{enumerate}
\item If $A$ is Hermitian, then $\forall x \in \mathbb{C}^n$, $y = x^H A x
  \in \mathbb{R}$.

\emph{Proof}: taking the hermitian transpose we have $y^H
  = x^H A^H x = x^H A x = y$, and the only scalars in $\mathbb{C}$
  that are equal to their own conjugates are the reals. $\Box$

\item If $A$ is Hermitian, and $\lambda$ is an eigenvalue of $A$, then
  $\lambda \in \mathbb{R}$. In particular, all eigenvalues of a
  symmetric real matrix are real (and so are the eigenvectors, since
  they are found by elimination on $A - \lambda I$, a real
  matrix).

\emph{Proof}: suppose $A x = \lambda x$ for some nonzero $x$, then
  pre-multiplying both sides by $x^H$, we get $x^H A x = x^H \lambda x
  = \lambda x^H x = \lambda \|x\|^2$, and since the left-hand side is
  real, and $\|x\|^2$ is real and positive, we conclude that $\lambda
  \in \mathbb{R}$. $\Box$

\item If $A$ is positive definite, and $\lambda$ is an eigenvalue of
  $A$, then $\lambda > 0$.

\emph{Proof}: Let nonzero $x$ be the
  eigenvector corresponding to $\lambda$. Then since $A$ is positive
  definite, we have $x^HAx > 0$ $\Longrightarrow$ $x^H (\lambda x) > 0$
  $\Longrightarrow$ $\lambda \|x\|^2 > 0$ $\Longrightarrow$ $\lambda >
  0$. $\Box$

\item If $A$ is Hermitian, and $x, y$ are the eigenvectors of $A$,
  corresponding to different eigenvalues $(Ax = \lambda_1 x, Ay =
  \lambda_2 y)$, then $x^H y = 0$.

\emph{Proof}: $\lambda_1 x^H y =
  (\lambda_1 x)^H y$ (since $\lambda_1$ is real) $= (Ax)^H y = x^H
  (A^H y) = x^H (A y) = x^H (\lambda_2 y) = \lambda_2 x^H y$, and get
  $(\lambda_1 - \lambda_2) x^H y = 0$. Since $\lambda_1 \neq
  \lambda_2$, we conclude that $x^H y = 0$. $\Box$


\item The above result means that if a real symmetric $n \times n$
  matrix $A$ has $n$ distinct eigenvalues, then the eigenvectors of
  $A$ are mutally orthogonal, and if we restrict ourselves to unit
  eigenvectors, we can decompose $A$ as $Q \Lambda Q^{-1}$, where
  $Q$ is orthogonal (why?), and therefore $A = Q \Lambda Q'$. We will
  later present the result that shows that it is true of EVERY
  symmetric matrix $A$ (whether or not it has $n$ distinct
  eigenvalues).


\item Unitary matrices preserve inner products and lengths.

\emph{Proof}: Let  $U$ be unitary. Then  $(Ux)^H (U y) = x^H U^H U y = x^H I y = x^H y$. In particular $\|Ux\| = \|x\|$. $\Box$

\item Let $U$ be unitary, and let $\lambda$ be an eigenvalue of
  $U$. Then $|\lambda| = 1$ (Note that $\lambda$ could be complex, for
  example $i$, or $\frac{1 + i}{\sqrt{2}}$).

\emph{Proof}: Suppose $Ux =  \lambda x$ for some nonzero $x$. Then $\|x\| = \|Ux\| = \|\lambda x\| = |\lambda| \|x\|$, and since $\|x\| > 0$, we have $|\lambda| =
  1$. $\Box$

\item Let $U$ be unitary, and let $x,y$ be eigenvectors of $U$,
  corresponding to different eigenvalues $(Ux = \lambda_1 x, Uy =
  \lambda_2 y)$. Then $x^H y = 0$.

\emph{Proof}: $x^H y = x^H I y = x^H U^H U
  y = (Ux)^H (Uy) = (\lambda_1 x)^H (\lambda_2 y) = \lambda_1^H
  \lambda_2 x^H y = \bar{\lambda_1} \lambda_2 x^H y$ (since
  $\lambda_1$ is a scalar). Suppose now that $x^H y \neq 0$, then
  $\bar{\lambda_1} \lambda_2 = 1$. But $|\lambda_1| = 1$
  $\Longrightarrow$ $\bar{\lambda_1} \lambda_1 = 1$, and we conclude
  that $\lambda_1 = \lambda_2$, a contradiction. Therefore, $x^H y =
  0$. $\Box$

\item For EVERY square matrix $A$, $\exists$ some unitary matrix $U$
  s.t. $U^{-1} A U = U^H A U = T$, where $T$ is upper triangular. We will not
  prove this result, but the proof can be found, for example, in
  section 5.6 of G.Strang's `Linear Algebra and Its Applications' (3rd
  ed.) This is a very important result which we're going to use in just
  a moment to prove the so-called Spectral Theorem.

\item If $A$ is normal, and $U$ is unitary, then $B = U^{-1} A U$ is
  normal.

\emph{Proof}: $B B^H = (U^H A U) (U^H A U)^H = U^H A U U^H A^H U =
  U^H A A^H U = U^H A^H A U$ (since $A$ is normal) $= U^H A^H U U^H A
  U = (U^H A U)^H (U^H A U) = B^H B$. $\Box$

\item If $n \times n$ $A$ is normal, then $\forall x \in
  \mathbb{C}^n$ we have $\|Ax\| = \|A^H x\|$.

\emph{Proof}: $\|Ax\|^2 = (Ax)^H
  Ax = x^H A^H A x = x^H A A^H x = (A^H x)^H (A^H x) = \|A^H
  x\|^2$. And since $\|Ax\| \geq 0 \leq \|A^H x\|$, we have $\|Ax\| =
  \|A^H x\|$. $\Box$

\item If $A$ is normal and $A$ is upper triangular, then $A$ is
  diagonal.

\emph{Proof}: Consider the first row of $A$. In the preceding
  result, let $x = \left[ \begin{array} {c} 1 \\ 0 \\ \vdots \\ 0
  \end{array} \right]$. Then $\|Ax\|^2 = |a_{11}|^2$(since the only
  non-zero entry in first column of $A$ is $a_{11}$) and $\|A^H x\|^2
  = |a_{11}|^2 + |a_{12}|^2 + \ldots + |a_{1n}|^2$. It follows
  immediately from the preceding result that $a_{12} = a_{13} = \ldots
  = a_{1n} = 0$, and the only non-zero entry in the first row of $A$
  is $a_{11}$. You can easily supply the proof that the only non-zero
  entry in the $i$-th row of $A$ is $a_{ii}$ and we conclude that $A$
  is diagonal. $\Box$

\item We have just succeded in proving the Spectral Theorem: If $A$ is
  $n \times n$ symmetric matrix, then we can write it as $A =
  Q \Lambda Q'$. We know that if $A$ is symmetric, then it's normal,
  and we know that we can find some unitary $U$ s.t. $U^{-1} A U = T$,
  where $T$ is upper triangular. But we know that $T$ is also normal,
  and being upper triangular, it is then diagonal. So $A$ is
  diagonalizable and by discussion above, the entries of $T = \Lambda$
  are eigenvalues of $A$ (and therefore real) and the columns of $U$
  are corresponding unit eigenvectors of $A$ (and therefore real), so
  $U$ is a real orthogonal matrix.\

\item More generally, we have shown that every normal matrix is
  diagonalizable.

\item If $A$ is positive definite, it has a square root $B$, s.t. $B^2
  = A$.

\emph{Proof}: We know that we can write $A = Q \Lambda Q'$, where all
  diagonal entries of $\Lambda$ are positive. Let $B = Q \Lambda^{1/2}
  Q'$, where $\Lambda^{1/2}$ is the diagonal matrix that has square
  roots of main diagonal elements of $\Lambda$ along its main
  diagonal, and calculate $B^2$ (more generally if $A$ is positive
  semi-definite, it has a square root). You should now prove for yourself
  that $A^{-1}$ is also positive definite and therefore $A^{-1/2}$
  also exists. $\Box$


\item If $A$ is idempotent, and $\lambda$ is an
  eigenvalue of $A$, then $\lambda = 1$ or $\lambda = 0$.

 \emph{Proof}: Exercise 6.4.

 %We know that $A = Q \Lambda Q'$ and $A^2 = A$, therefore $Q \Lambda Q'
 %Q \Lambda Q' = Q \Lambda^2 Q' = Q \Lambda Q'$, and we conclude that
 %$\Lambda = \Lambda^2$. You should prove for yourself that this
  %implies that diagonal entries of $\Lambda$ are either 0 or 1, and
  %that the number of 1's along the main diagonal of $\Lambda$ =
  %$rank(A)$. Why is this another proof that $rank(A) = tr(A)$ for
  %symmetric and idempotent $A$?
\end{enumerate}

There is another way to think about the result of the Spectral
  theorem. Let $x \in \mathbb{R}^n$ and consider $Ax = Q \Lambda Q'
  x$. Then (do it as an exercise!) carrying out the matrix
  multiplication on $Q \Lambda Q'$ and letting $q_1, q_2, \ldots, q_n$
  denote the columns of $Q$ and $\lambda_1, \lambda_2, \ldots,
  \lambda_n$ denote the diagonal entries of $\Lambda$, we have: $Q
  \Lambda Q'= \lambda_1 q_1 q_1' + \lambda_2 q_2 q_2' + \ldots +
  \lambda_n q_n q_n'$ and so $Ax = \lambda_1 q_1 q_1' x + \lambda_2
  q_2 q_2' x + \ldots + \lambda_n q_n q_n' x$. We recognize $q_i q_i'$
  as the projection matrix onto the line spanned by $q_i$, and thus
  every $n \times n$ symmetric matrix is the sum of $n$ 1-dimensional
  projections. That should come as no surprise: we have orthonormal
  basis $q_1, q_2, \ldots q_n$ for $\mathbb{R}^n$, therefore we can
  write every $x \in \mathbb{R}^n$ as a unique combination $c_1 q_1 +
  c_2 q_2 + \ldots + c_n q_n$, where $c_1 q_1$ is precisely the
  projection of $x$ onto line through $q_1$. Then applying $A$ to the
  expression we have $Ax = \lambda_1 c_1 q_1 + \lambda_2 c_2 q_2 +
  \ldots + \lambda_n c_n q_n$, which of course is just the same thing
  as we have above.


\subsection{SVD and Pseudo-inverse}
\textbf{Theorem}: Every $m \times n$  matrix $A$ can be written as $A
= Q_1 \Sigma Q_2'$, where $Q_1$ is $m \times m$ orthogonal, $\Sigma$ is
$m \times n$ pseudo-diagonal (meaning that that the first $r$
diagonal entries $\sigma_{ii}$ are non-zero and the rest of the matrix
entries are zero, where $r = rank(A)$), and $Q_2$ is $n \times n$
orthogonal. Moreover, the first $r$ columns of $Q_1$ form an
orthonormal basis for $col(A)$, the last $m-r$ columns of $Q_1$ form
an orthonormal basis for $N(A')$, the first $r$ columns of $Q_2$ form
an orthonormal basis for $col(A')$, last $n-r$ columns of $Q_2$ form
an orthonormal basis for $N(A)$, and the non-zero entries of $\Sigma$
are the square roots of non-zero eigenvalues of both $AA'$ and
$A'A$. (It is a good exersise at this point for you to prove that
$AA'$ and $A'A$ do in fact have same eigenvalues. What is the
relationship between eigenvectors?). This is known as the \textbf{Singular
  Value Decomposition} or SVD.\\

\noindent \textbf{Proof}: $A'A$ is $n \times n$ symmetric and therefore has a
set of $n$ real orthonormal eigenvectors. Since $rank(A'A) = rank(A) =
r$, we can see that $A'A$ has $r$ non-zero (possibly-repeated)
eigenvalues (Exercise 6.3). Arrange the eigenvectors $x_1, x_2, \ldots, x_n$
in such a way that the first $x_1, x_2, \ldots, x_r$ correspond to
non-zero $\lambda_1, \lambda_2, \ldots, \lambda_r$ and put $x_1, x_2,
\ldots, x_n$ as columns of $Q_2$.  Note that as $x_{r+1}, x_{r + 2}, \ldots, x_n$ form a basis for $N(A)$ by Exercise 2.4 as they are linearly independent, $dim(N(A)) = n- r$ and \[x_i \in N(A) \hspace{10 pt} \hspace{10 pt} for \hspace{10 pt} i=r+1,...,n. \]
Therefore $x_1, x_2, \ldots, x_r$ form a basis for the row
space of $A$. Now set $\sigma_{ii} = \sqrt{\lambda_i}$ for $1 \leq i
\leq r$, and let the rest of the entries of $m \times n$ $\Sigma$ be
0. Finally, for $1 \leq i \leq r$, let $q_i =
\frac{Ax_i}{\sigma_{ii}}$. You should verify for yourself that $q_i$'s
are orthonormal ($q_i ' q_j = 0$ if $i \neq j$, and $q_i' q_i =
1$). By Gram-Schmidt, we can extend the set $q_1, q_2, \ldots, q_r$ to
a complete orthonormal basis for $\mathbb{R}^m$, $q_1, q_2, \ldots,
q_r, q_{r+1}, \ldots, q_n$. As $q_1, q_2, \ldots, q_r$ are each in the column space of $A$ and linearly independent, they form an orthonormal basis for column space of $A$ and therefore $q_{r+1}, q_{r+2}, \ldots, q_n$ form an orthonormal
basis for the left nullspace of $A$. We now verify that $A = Q_1 \Sigma
Q_2 '$ by checking that $Q_1' A Q_2 = \Sigma$. Consider $ij$-th entry
of $Q_1 ' A Q_2$. It is equal to $q_i ' Ax_j$. For $j > r$,
$Ax_j = 0$ (why?), and for $j \leq r$ the expresison becomes $q_i '
\sigma_{jj} q_j = \sigma_{jj} q_i ' q_j = 0$(if $i \neq j)$ or $1$ (if
$i = j$). And therefore $Q_1 ' A Q_2 = \Sigma$, as claimed. $\Box$\\


One important application of this decomposition is in estimating
$\beta$ in the system we had before when the columns of $X$ are
linearly dependent. Then $X'X$ is not invertible, and more than one
value of $\hat{\beta}$ will result in $X'(Y-X \hat{\beta}) = 0$. By
convention, in cases like this, we choose $\hat{\beta}$ that has the
smallest length. For example, if both $\left[ \begin{array} {c} 1 \\ 1
    \\ 1 \end{array} \right]$ and $\left[ \begin{array} {c} 1 \\ 1
    \\ 0 \end{array} \right]$ satisfy the normal equations, then we'll choose
  the latter and not the former. This optimal value of $\hat{\beta}$ is
  given by $\hat{\beta} = X^+ Y$, where $X^+$ is a $p \times n$ matrix
  defined as follows: suppose $X$ has rank $r < p$ and it has
  S.V.D. $Q_1 \Sigma Q_2 '$. Then $X^+ = Q_2 \Sigma^+ Q_1'$, where
  $\Sigma^+$ is $p \times n$ matrix s.t. for $1 \leq i \leq r$
  we let ${\sigma^+}_{ii} = 1/{\sigma_{ii}}$ and ${\sigma^+}_{ij} = 0$
  otherwise. We will not prove this fact, but the proof can be found
  (among other places) in appendix 1 of Strang's book. The matrix $X^+ $ is called the \textbf{pseudo-inverse} of the matrix $X$. 
  The pseudo-inverse is defined and unique for all matrices whose entries are real or complex numbers. 

\subsection*{Exercises}
\addcontentsline{toc}{subsection}{Exercises}

\begin{exr}{}
Show that if a matrix is triangular, its determinant is the product of the entries on the main diagonal.
\end{exr}

\begin{exr}{}
Let $s_1, s_2, \ldots, s_n \in {\mathbb{R}}^n$ be
the set of linearly independent eigenvectors of $A$, let $\lambda_1,
\lambda_2, \ldots, \lambda_n$ be the corresponding eigenvalues (note that
they need not be distinct), and let $S$ be the $n \times n$ matrix such that the $j$-th
column of which is $s_j$. Show that if $\Lambda$ is the $n \times n$
diagonal matrix s.t. the $ii$-th entry on the main diagonal is
$\lambda_i$, then $AS = S \Lambda$, and since $S$ is invertible (why?) we
have $S^{-1} A S = \Lambda$.
\end{exr}

\begin{exr}{}
Show that if $rank(A) = r$, then $A'A$ has $r$ non-zero eigenvalues.
\end{exr}

\begin{exr}{}
Show that if $A$ is idempotent, and $\lambda$ is an
  eigenvalue of $A$, then $\lambda = 1$ or $\lambda = 0$.
\end{exr}

\clearpage
\newpage
\section{Statistics: Random Variables}
This sections covers some basic properties of random variables.  While this material is not necessarily tied directly to linear algebra, it is essential background for graduate level Statistics, O.R., and Biostatistics.  For further review of these concepts, see Casella and Berger, sections 2.1, 2.2, 2.3, 3.1, 3.2, 3.3, 4.1, 4.2, 4.5, and 4.6.

Much of this section is gratefully adapted from Andrew Nobel's lecture notes.

\subsection{Expectation, Variance and Covariance}
The \textbf{expected value} of a continuous random variable $X$, with probability density function $f$, is defined by
\[EX = \int_{-\infty}^{\infty} x f(x) dx. \]
The expected value of a discrete random variable $X$, with probability mass function $p$, is defined by
\[EX = \sum_{x \in \mathbb{R}, p(x) \neq 0} x p(x).\]
The expected value is \textbf{well-defined} if $E|X| < \infty$. \\

\noindent We now list some basic properties of $E(\cdot)$:
\begin{enumerate}
\item  $X \leq Y$ imples $EX \leq EY$

\emph{Proof:}  Follows directly from properties of $\int$ and $\sum$.

\item For $a,b \in \mathbb{R}$, $E(aX + bY) = a EX + b EY$.

\emph{Proof:}  Follows directly from properties of $\int$ and $\sum$.

\item $|EX| \leq E|X|$

\emph{Proof:} Note that $X, -X \leq |X|$.  Hence, $EX, -EX \leq E|X|$ and therefore $|EX| \leq E|X|$. $\Box$

\item If $X$ and $Y$ are independent ($X \coprod Y$), then $E(XY) = EX \cdot EY$.

\emph{Proof:} See Theorem 4.2.10 in Casella and Berger.

\item If $X$ is a non-negative continuous random variable, then
 $$EX = \int_0^\infty P(X \geq t) dt = \int_0^\infty (1-F(t)) dt.$$

\emph{Proof:} Suppose $X \sim f$.  Then,
\begin{eqnarray*} \int_0^\infty P(X>t) dt &=& \int_0^\infty [\int_t^\infty f(x) dx] dt \\
	&=& \int_0^\infty [ \int_0^\infty f(x) I(x>t) dx] dt \\
	&=&  \int_0^\infty \int_0^\infty f(x) I(x>t) dt dx \hspace{12 pt} \mbox{(Fubini)} \\
	&=& \int_0^\infty f(x) [ \int_0^\infty I(x>t) dt] dx \\
	&=& \int_0^\infty x f(x) dx = EX \hspace{70 pt}\Box \end{eqnarray*}
	
\item If $X \sim f$ then $Eg(X) = \int g(x) f(x) dx$. \\
If $X \sim p$ then $Eg(x)  = \underset{x}{\sum} g(x) p(x) $.
	
\emph{Proof:} Follows from definition of $Eg(X)$.
\end{enumerate}

\noindent The \textbf{variance} of a random variable $X$ is defined by
\begin{eqnarray*} \mbox{Var}(X) &=& E(X-EX)^2 \\ &=& EX^2 - (EX)^2 . \end{eqnarray*}
Note that $\mbox{Var}(X)$ is finite (and therefore well-defined) if $EX^2 < \infty$.
The \textbf{covariance} of two random variables $X$ and $Y$ is defined by
\begin{eqnarray*} \mbox{Cov}(X,Y) &=& E[(X-EX)(Y-EY)] \\ &=& E(XY) - EXEY. \end{eqnarray*}
Note that $\mbox{Cov}(X,Y) $ is finite if $EX^2, EY^2 < \infty$. \\

\noindent We now list some general properties, that follow from the definition of variance and covariance:
\begin{enumerate}
\item $\mbox{Var}(X) \geq 0$, with ``$=$'' iff $X$ is constant with probability 1.
\item For $a,b \in \mathbb{R}$, $\mbox{Var}(aX+b) = a^2 \mbox{Var}(X$).
\item If $X \coprod Y$, then $\mbox{Cov}(X,Y) = 0$.  The converse, however, is not true in general.
\item $\mbox{Cov}(aX +b,cY+d) = ac \mbox{Cov}(X,Y)$.
\item If $X_1,\hdots,X_n$ satisfy $EX_i^2 < \infty$, then
\[\mbox{Var}(\displaystyle\sum\limits_{i=1}^n X_i) = \displaystyle\sum\limits_{i=1}^n \mbox{Var}(X_i) + 2 \displaystyle\sum\limits_{i<j} \mbox{Cov}(X_i,X_j).\]
\end{enumerate}

\subsection{Distribution of Functions of Random Variables}

Here we describe various methods to calculate the distribution of a function of one or more random variables. \\

\noindent \emph{CDF method}\\

 For the single variable case, given $X \sim f_X$ and $g: \mathbb{R} \rightarrow \mathbb{R}$ we would like to find the density of $Y=g(X)$, if it exists.  A straightforward approach is the \textbf{CDF method}:
\begin{itemize}
\item Find $F_Y$ in terms of $F_X$
\item Differentiate $F_Y$ to get $f_Y$
\end{itemize}

\noindent \textbf{Example 1: Location and scale.}  Let $X \sim f_X$ and $Y = aX+b$, with $a>0$.  Then,
\begin{eqnarray*} F_Y(y) &=& P(Y \leq y) = P(aX+b \leq y)  = P(X \leq \frac{y-b}{a}) \\ &=& F_X (\frac{y-b}{a}). \end{eqnarray*}
Thus, $f_Y (y) = F'_Y (y) = a^{-1} f_X(\frac{y-b}{a}).$

If $a<0$, a similar argument shows $f_Y(y) = |a|^{-1} f(\frac{y-b}{a})$.  \\

\noindent \textbf {Example 2} If $X \sim \mathbb{N}(0,1)$ and Y = aX + b, then
\begin{eqnarray*} f_Y(y) &=& |a|^{-1} \phi (\frac{y-b}{a}) \\ &=& \frac{1}{\sqrt{2 \pi a^2}} \mbox{exp} \left\{ -\frac{(y-b)^2}{2a^2} \right\} \\ &=&   \mathbb{N}(b,a^2). \end{eqnarray*}

\noindent \textbf{Example 3} Suppose $X \sim \mathbb{N}(0,1)$.  Let $Z = X^2$.  Then,
\begin{eqnarray*}  F_Z (z) &=& P(Z \leq z) = P(X^2 \leq z) = P(-\sqrt{z} \leq X \leq \sqrt{z}) \\ &=& \Phi(\sqrt{z}) - \Phi(-\sqrt{z}) = 1-2\Phi(-\sqrt{z}). \end{eqnarray*}
Thus, $f_Z(z) = z^{-1/2} \phi (-\sqrt{z}) = \frac{1}{\sqrt{2 \pi}} z^{-1/2} e^{-z/2}$. \\

\noindent \emph{Convolutions}\\

The \textbf{convolution} $f=f_1 \ast f_2$ of two densities $f_1$ and $f_2$ is defined by \[f(x) = \int_{-\infty}^{\infty} f_1(x-y)f_2(y)dy.\]
Note that $f(x) \geq 0$, and
\begin{align*}
\int_{-\infty}^{\infty} f(x) dx &= \int_{-\infty}^{\infty} \left[\int_{-\infty}^{\infty}  f_1(x-y)f_2(y)dy\right]dx \\
&=  \int_{-\infty}^{\infty} \int_{-\infty}^{\infty} f_1(x-y)f_2(y)d_xd_y\\
&= \int_{-\infty}^{\infty} f_2(y) \left[\int_{-\infty}^{\infty} f_1 (x-y)dx \right] dy = \int_{-\infty}^{\infty}f_2(y)dy = 1.
\end{align*}
So, $f = f_1 \ast f_2$ is a density. \\

\noindent \textbf{Theorem:}  If $X \sim f_X$, and $Y \sim f_Y$ and X and Y are independent, then $X+Y \sim f_X \ast f_Y$.\\

\noindent \emph{Proof:} Note that
\begin{align*}
P(X+Y \leq v) &= \int_{-\infty}^\infty  \int_{-\infty}^\infty  f_X(x) f_Y(y)  I\{ (x,y):x+y \leq v \} dx dy \\
&= \int_{-\infty}^\infty \int_{-\infty}^{v-y} f_X(x)f_Y(y)dxdy  \\
&=  \int_{-\infty}^\infty\left[ \int_{-\infty}^{v-y} f_X(x) dx\right] f_Y(y) dy \\
&= \int_{-\infty}^\infty \left[ \int_{-\infty}^{v} f_X (u-y) du\right] f_Y(y) dy \hspace{20 pt} (u = y+x) \\
&= \int_{-\infty}^v \left[\int_{-\infty}^{\infty} f_X(u-y)f_Y(y)dy\right]du \\
&= \int_{-\infty}^v (f_X \ast f_Y)(u) du . \hspace{30 pt}\Box
\end{align*}

\noindent \textbf{Corollary:} Convolutions are commutative and associative.  If $f_1,f_2,f_3$ are densities, then
\begin{align*}
f_1 \ast f_2 &= f_2 \ast f_1 \\
(f_1 \ast f_2) \ast f_3 &= f_1 \ast (f_2 \ast f_3).
\end{align*}




\noindent \emph{Change of Variables}\\

We now consider functions of more than one random variable.  In particular, let $U,V$ be open subsets in $\mathbb{R}^k$, and $H: U \rightarrow V$.  Then, if $\vec{x}$ is a vector in $U$,  \[ H(\vec{x}) = (h_1(\vec{x}) , \hdots,h_k(\vec{x}))^t. \]
is a vector in $V$.  The functions $h_1(\cdot), \hdots,h_k(\cdot)$ are the \textbf{coordinate functions} of $H$.  If $\vec{X}$ is a continuous random vector, we would like to find the density of $H(\vec{X})$. First, some further assumptions:

\begin{Assumptions}
\item $H: U \rightarrow V$ is one-to-one and onto.
\item $H$ is continuous.
\item For every $1 \leq i,j \leq k$, the partial derivatives \[h'_{ij} \equiv \frac{\partial h_i}{\partial x_j}\] exist and are continuous.
\end{Assumptions}

\noindent Let $D_H(\vec{x})$ be the matrix of partial derivatives of $H$:
\[D_H(\vec{x}) = [h'_{ij}(\vec{x}) : 1 \leq i,j \leq k] .\]
Then, the \textbf{Jacobian} (or \textbf{Jacobian determinant}\footnote{The partial derivative matrix $D$ is sometimes called the \textbf{Jacobain matrix} (see Section \ref{Sec5.2}).}) of $H$ at $\vec{x}$ is the determinant of $D_H(\vec{x})$: \[J_H(\vec{x}) = \mbox{det}(D_H(\vec{x})). \]
The assumptions \textbf{A1-3} imply that $H^{-1}: V \rightarrow U$ exists and is differentiable on $V$ with
\[J_{H^{-1}} (\vec{y}) = (J_H(H^{-1}(y)))^{-1}.\] \\

\noindent \textbf{Theorem:} Suppose $J_H(\vec{x}) \neq 0$ on $U$.  If $\vec{X} \sim f_{\vec{X}}$ is a k-dimensional random vector such that $P(\vec{X} \in U) = 1$, then $\vec{Y} = H(\vec{X})$ has density
\begin{eqnarray*}
f_{\vec{Y}}(\vec{y}) &=& f_{\vec{X}} (H^{-1}(\vec{y}))\cdot|J_{H^{-1}}(\vec{y})| \\ &=& f_{\vec{X}} (H^{-1}(\vec{y})) \cdot |J_H(H^{-1}(\vec{y}))|^{-1} .
\end{eqnarray*}

\noindent \textbf{Example:} Suppose $X_1,X_2$ are jointly continuous with density $f_{X_1,X_2}$.  Let $Y_1 = X_1+X_2$, $Y_2 = X_1-X_2$,  and find $f_{Y_1,Y_2}$.

Here \begin{eqnarray*} y_1 = h_1(x_1,x_2) &=& x_1+x_2 \\
			       y_2 = h_2(x_1,x_2) &=& x_1-x_2 \\
			       x_1 = g_1(y_1,y_2) &=& \frac{1}{2} (y_1+y_2) \\
			       x_2 = g_2(y_1,y_2) &=& \frac{1}{2} (y_1-y_2) , \end{eqnarray*}
and
\[J_H(x_1,x_2) = \left | \begin{array}{cc} \frac{\partial h_1}{\partial x_1} & \frac{\partial h_1}{\partial x_2} \\ \frac{\partial h_2}{\partial x_1} & \frac{\partial h_2}{\partial x_2} \end{array} \right |
		= \left | \begin{array}{cc} 1 & 1 \\ 1 & -1 \end{array} \right | = -2 \neq 0. \]
So, applying the theorem, we get
\[f_{Y_1,Y_2}(y_1,y_2) = \frac{1}{2} f_{X_1,X_2} (\frac{y_1+y_2}{2},\frac{y_1-y_2}{2}). \] \\

As a special case, assume $X_1,X_2$ are $\mathbb{N}(0,1)$ and independent.  Then,

\begin{eqnarray*} f_{Y_1,Y_2}(y_1,y_2) &=& \frac{1}{2} \phi(\frac{y_1+y_2}{2}) \phi(\frac{y_1-y_2}{2}) \\
					&=& \frac{1}{4 \pi} \mbox{exp} \left \{-\frac{(y_1+y_2)^2}{8} - \frac{(y_1-y_2)^2}{8} \right \} \\
					&=&  \frac{1}{4 \pi} \mbox{exp} \left \{-\frac{2y_1^2+2y_2^2}{8} \right \} \\
					&=&  \frac{1}{4 \pi} \mbox{exp} \left \{-\frac{y_1^2}{4} \right \} \mbox{exp} \left \{-\frac{y_2^2}{4} \right \}. \end{eqnarray*}

So, both $Y_1$ and $Y_2$ are $\mathbb{N}(0,2)$, and they are independent!

\subsection{Derivation of Common Univariate Distributions}

\noindent \emph{Double Exponential} \\

If $X_1, X_2 \sim \mbox{Exp} (\lambda)$ and $X_1 \coprod X_2$, then $X_1 - X_2$ has a \textbf{double exponential} (or \textbf{Laplace}) distribution:  $X_1 - X_2 \sim \mbox{DE}(\lambda)$.  The density of $\mbox{DE}(\lambda)$,
\[f(x) = \frac{\lambda}{2} e^{-\lambda |x|} \hspace{20 pt} -\infty < x < \infty ,\]
can be derived through the convolution formula. \\

\noindent \emph{Gamma and Beta Distributions}\\

The \textbf{gamma function}, a component in several probability distributions,  is defined by \[\Gamma (t)  = \int_0^\infty x^{t-1}e^{-x} dx, \hspace{20 pt} t > 0.\]
Here are some basic properties of $\Gamma(\cdot):$
\begin{enumerate}
\item $\Gamma(t)$ is well-defined for $t>0$.

\emph{Proof:} For $t>0$, \[0 \leq \Gamma(t) \leq \int_0^1 x^{t-1} dx + \int_1^\infty x^{t-1}e^{-x}dx < \infty. \hspace{20 pt} \Box \]

\item $\Gamma (1) = 1$.

\emph{Proof:} Clear.

\item $\forall x>0,$ $\Gamma(x+1) = x \Gamma(x)$.

\emph{Proof:} Exercise 7.4.

\item $\Gamma (n+1) = n!$ for $n=0,1,2,...$.

\emph{Proof:} Follows from $2,3$.

\item log $\Gamma(\cdot)$ is convex on $[0,\infty)$.\\
\end{enumerate}

\noindent The \textbf{gamma distribution} with parameters $\alpha, \beta > 0$, $\Gamma(\alpha,\beta)$, has density
\[g_{\alpha, \beta} (x) = \frac{\beta^{\alpha} x^{\alpha-1} e^{-\beta x}}{\Gamma (\alpha)}, \hspace{20 pt} x>0.\]

\noindent Note: A basic change of variables shows that for $s>0$, \[X \sim \Gamma(\alpha, \beta) \iff sX \sim \Gamma \left(\alpha, \frac{\beta}{s} \right). \]
So, $\beta$ acts as a scale parameter of the $\Gamma(\alpha, \cdot)$ family.  The parameter $\alpha$ controls shape:
\begin{itemize}
\item If $0 < \alpha < 1$, then $g_{\alpha, \beta} (\cdot)$ is convex and $g_{\alpha,\beta} \uparrow \infty$ as $x \rightarrow 0$.
\item If $\alpha > 1$, then $g_{\alpha, \beta} (\cdot)$ is unimodal, with maximum at $x = \frac{\alpha-1}{\beta}$.
\end{itemize}

\noindent If $X \sim \Gamma(\alpha, \beta)$, then $EX = \frac{\alpha}{\beta}$, $\mbox{Var}(X) = \frac{\alpha}{\beta^2}$ .

We now use convolutions to show that if $X \sim \Gamma(\alpha_1,\beta), Y\sim \Gamma(\alpha_2,\beta)$ are independent then $X+Y \sim \Gamma(\alpha_1+\alpha_2,\beta)$: \\

\noindent \textbf{Theorem:} The family of distributions $\{\Gamma(\cdot,\beta)\}$ is closed under convolutions.  In particular
\[ \Gamma (\alpha_1,\beta) \ast \Gamma(\alpha_2, \beta) = \Gamma(\alpha_1+\alpha_2,\beta).\]

\noindent \emph{Proof:} For $x>0$,
\begin{align}f(x) &= (g_{\alpha_1,\beta} \ast g_{\alpha_2,\beta}) (x) \nonumber \\
&= \int_0^x g_{\alpha_1,\beta}(x-u) g_{\alpha_2,\beta} (u) du \nonumber \\
&= \frac{\beta^{\alpha_1+\alpha_2}}{\Gamma (\alpha_1) \Gamma (\alpha_2)} e^{-\beta x} \int_0^x (x-u)^{\alpha_1-1} u^{\alpha_2-1} du \label{GamConv}\\
&= \mbox{const} \cdot e^{-\beta x} x^{\alpha_1+\alpha_2 -1} \nonumber
\end{align}
Thus, $f(x)$ and $g_{\alpha_1+\alpha_2, \beta}(x)$ agree up to constants.  As both integrate to 1, they are the same function. $\Box$\\

\noindent \textbf{Corollary:}  Note if $\alpha = 1$, then $\Gamma (1,\beta) = \mbox{Exp} (\beta)$.  Hence, If $X_1,\hdots,X_n$ are iid $\sim \mbox{Exp}(\lambda)$, then \[Y=X_1+\hdots+X_n \sim \Gamma(n,\lambda),\] with density
\[f_Y(y) = \frac{\lambda^2 y^{n-1} e^{-\lambda y}}{(n-1)!} .\]
This is also known as an \textbf{Erlang} distribution with parameters $n$ and $\lambda$.\\

It follow from equation (\ref{GamConv}), with $x=1$ that
\begin{eqnarray*}
\frac{\beta^{\alpha_1+\alpha_2}}{\Gamma (\alpha_1) \Gamma (\alpha_2)} e^{-\beta} \int_0^1 (1-u)^{\alpha_1-1} u^{\alpha_2-1} du \\
= g_{\alpha_1+\alpha_2,\beta}(1) = \frac{\beta^{\alpha_1+\alpha_2}e^{-\beta}}{\Gamma (\alpha_1+\alpha_2)} .\end{eqnarray*}
Rearranging terms shows that for $r,s>0$,
\[B(r,s)=\frac{\Gamma (r) \Gamma(s)}{\Gamma(r+s)} = \int_0^1 (1-u)^{r-1} u^{s-1} du. \]
Here $B(\cdot,\cdot)$ is known as the \textbf{beta function} with parameters $r,s$.  The \textbf{beta distribution} $Beta(r,s)$ has density
\[b_{r,s}(x) = B(r,s)^{-1} \cdot x^{r-1}(1-x)^{s-1}, \hspace{20 pt} 0<x<1.\]
The parameters $r,s$ play symmetric roles.  If $r=s$ then $Beta (r,s)$ is symmetric about $1/2$.  $Beta(r,r)$ is u-shaped if $r < 1$, uniform if $r=1$, and unimodal (bell shaped) if $r>1$.  If $r>s>0$ then $Beta(r,s)$ is skewed to the right, if $0<s<r$ then $Beta(r,s)$ is skewed left.  The random variable $X \sim Beta(r,s)$ has expection and variance
\[EX = \frac{r}{r+s}, \hspace{20 pt} \mbox{Var}(X) = \frac{rs}{(r+s)^2(r+s+1)}.\] \\

\noindent \emph{Chi-square distributions}\\

\noindent Fix an integer $k \geq 1$.  Then, the chi-square distribution with $k$ degrees of freedom, written $\chi^2_k$, is $\Gamma(k/2,1/2)$.  Thus, $\chi^2_k$ has density
\[f_k(x) = \frac{1}{2^{k/2} \Gamma(\frac{k}{2})} x^{\frac{k}{2}-1}e^{-\frac{x}{2}} , \hspace{20 pt} x>0.\]

\noindent \textbf{Theorem:} If $X_1,\hdots,X_k$ are iid $\mathbb{N}(0,1)$, then $X_1^2+\hdots+X_k^2 \sim \chi_k^2$.\\

\noindent\emph{Proof:}  Recall that if $X \sim \mathbb{N}(0,1)$ then $X^2 \sim f(x) = \frac{1}{2 \sqrt{\pi}} e^{-\frac{x}{2}} = \Gamma(\frac{1}{2},\frac{1}{2})$.  Thus, $X^2 \sim \chi_1^2$.  Furthermore,
\[X_1^2+\hdots+X_k^2 \sim \Gamma\left(\frac{k}{2},\frac{1}{2}\right) = \chi_k^2 . \hspace{20 pt} \Box \]

\noindent If $Y=X_1^2+\hdots+X_k^2 \sim \chi_k^2$, then
\begin{align}
E Y= E(X_1^2+\hdots+X_k^2) = k EX_1^2 = k. \nonumber
\end{align}
\begin{align}
\mbox{Var}(Y) &= k \mbox{Var}(X_1^2) = k(E X_1^4-(EX_1^2)^2) \nonumber \\
&= k(3-1) = 2k \nonumber.
\end{align}

\noindent \emph{F and t-distributions}\\

\noindent The \textbf{F-distribution} with with m,n degrees of freedom, $F(m,n)$, is the distribution of the ratio
\[\frac{X/m}{Y/n},\]
where $X \sim \chi_m^2$, $Y \sim \chi_n^2$, and $X \coprod Y$. \\

%\noindent \textbf{Fact:} If $X \sim f_x$ and $0 < Y \sim f_Y$ then $R = \frac{X}{Y}$.\\

%\noindent \emph{Proof:} Use CDF method.\\

%\noindent By the fact above, $F(m,n)$ has density

The density of $F(m,n)$  is
\[f_{m,n}(x) = B^{-1} \left(\frac{m}{2}, \frac{n}{2}\right) \left(\frac{m}{n}\right)^{\frac{m}{2}} x^{\frac{m}{2}-1}\left(1+\frac{m}{n}x\right)^{-\frac{1}{2} (m+n)}. \]

\noindent The \textbf{t-distribution} with $n$ degrees of freedom, $t_n$, is the distribution of the ratio
\[\frac{X}{\sqrt{Y^2/n}},\]
where $X,Y \sim \mathbb{N}(0,1)$ are independent.  Equivalently, $t_n$ is the distribution of $\sqrt{Z}$ where $Z \sim F(1,n)$. The density of $t_n$ is
\[f_n(t) = \frac{1}{nB\left(\frac{1}{2},\frac{n}{2}\right)} \cdot \left(1+\frac{t^2}{n}\right)^{-(n+1)/2}. \]
Some other properties of the t-distribution:
\begin{enumerate}
\item $t_1$ is the \textbf{Cauchy distribution}.
\item If $X \sim t_n$ then $EX = 0$ for $n \geq 2$, undefined for $n=1$; $\mbox{Var}(X) = \frac{n}{n-2}$ for $n \geq 3$, undefined for $n=1,2$.
\item The density $f_n(t)$ converges to the density of a standard normal, $\phi(t)$, as $n \rightarrow \infty.$
\end{enumerate}


\subsection{Random Vectors: Expectation and Variance}
A \textbf{random vector} is a vector $X = [X_1 \, X_2 \, ... X_k]'$ whose components $X_1, X_2,...,X_k$ are real-valued random variables defined on the same probability space.  The expectation of a random vector $E(X)$, if it exists, is given by the expected value of each component:

\[E(X) = [EX_1 \, EX_2 \, ... \, EX_k]'.\]

The covariance matrix of a random vector $Cov(X)$ is given by

\[Cov(X) = E[(X-EX)(X-EX)']=E(XX')-EXEX'.\]



We now give some general results on expectations and variances. We supply
reasonings for some of them, and you should verify the rest (usually
by the method of entry-by-entry comparison). We assume in what follows
that $k \times k$ $A$ and $k \times 1$ $a$ are constant, and we let
$k \times 1$ $ \mu = E(X)$ and $k \times k$ $V = Cov(X)$ ($v_{ij} =
Cov(X_i, X_j)$):

\begin{enumerate}
\item $E(AX) = A E(X)$

\emph{Proof}: Exercise 7.5(a).

\item $Var(a'X) = a'Va$.

\emph{Proof}: Note that  \begin{eqnarray*} var(a'X) &=& var(a_1 X_1 + a_2 X_2 + \ldots + a_k X_k) \\ &=& \displaystyle\sum_{i=1}^k {\displaystyle\sum_{j=1}^k {a_i a_j
  Cov(X_i X_j)}}  \\ &=& \displaystyle\sum_{i=1}^k
  {\displaystyle\sum_{j=1}^k {v_{ij} a_i a_j}} = a'Va \hspace{10 pt} \Box \end{eqnarray*}

\item $Cov(AX) = AVA'$

\emph{Proof} Exercise 7.5(b).

\item $E(X'AX) = tr(AV) + \mu' A \mu$

\emph{Proof}: Let $A_i$ be the $i$th row of A and $a_{ij}$ be the $ij$th entry of A. \\
Note that 
$tr(AV)=tr(A(E(XX')-EXEX'))=tr(AE(XX'))-tr(AEXEX'). $
\begin{eqnarray*}
tr(AE(XX'))&=& tr\left(\begin{pmatrix} A_1E(XX')\\ \vdots \\ A_kE(XX')  \end{pmatrix}  \right)\\
	&=& \sum_{i=1}^k\sum_{j=1}^k a_{ij}E(X_jX_i)\\
	&=&E\left(\sum_{i=1}^k\sum_{j=1}^k a_{ij}X_jX_i \right)\\
	&=&E\left(\sum_{i=1}^k A_iXX_i\right)\\
	&=&E\left(\left(\sum_{i=1}^k X_iA_i\right)X\right)\\
	&=&E(X'AX) 
\end{eqnarray*}
Meanwhile, 
$$tr(AEXEX')=tr(EX'AEX)=EX'AEX=\mu' A\mu. $$
So we have $E(X'AX) = tr(AV) + \mu' A \mu. \hspace{10 pt} \Box$


\item Covariance matrix $V$ is positive semi-definite.

\emph{Proof}: $y'Vy =  Var(y'X) \geq 0 $ $\forall y \neq 0$. Since $V$ is symmetric (why?), it follows that $V^{1/2} = (V^{1/2})'$. $\Box$

\item $Cov(a'X, b'X) = a'Vb$

\emph{Proof}: Exercise 7.5(c).

\item If $X,Y$ are two $k \times 1$ vectors of random variables, we
  define their \textbf{cross-covariance} matrix $C$ as follows :
  $c_{ij} = Cov(X_i, Y_j)$. Notice that unlike usual covariance
  matrices, a cross-covariance matrix is not (usually) symmetric. We
  still use the notation $Cov(X,Y)$ and the meaning should be clear
  from the context. Now, suppose $A, B$ are $k \times k$. Then
  $Cov(AX, BX) = A V B'$.
  
  \emph{Proof}: Let $c_{ij}$ be the $ij$th entry of $Cov(AX, BX)$. Denote the $i$th row vectors of $A$ and $B$ as $A_i$ and $B_i$, respectively. By the result above,
   $$c_{ij} = A_iVB_j = ij \text{th entry of } A V B'. \hspace{10 pt} \Box$$
\end{enumerate}



 \subsection*{Exercises}
\addcontentsline{toc}{subsection}{Exercises}

\begin{exr}{}
Show that if $X \sim f$ and $g(\cdot)$ is non-negative, then $Eg(X) = \int_{-\infty}^\infty g(x) f(x) dx$.

\noindent[Hint: Recall that $EX = \int_0^\infty P(X > t) dt$ if $X \geq 0$.]
\end{exr}

\begin{exr}{}
Let $X$ be a continuous random variable with density $f_X$.  Find the density of $Y = |X|$ in terms of $f_X$.
\end{exr}


\begin{exr}{}
Let $X_1 \sim \Gamma (\alpha_1, 1)$ and $X_2 \sim \Gamma (\alpha_2,1)$ be indepedent.  Use the two-dimensional change of variables formula to show that $Y_1 = X_1+X_2$ and $Y_2 = X_1/(X_1 + X_2)$ are independent with $Y_1 \sim \Gamma(\alpha_1+\alpha_2,1)$ and $Y_2 \sim Beta(\alpha_1,\alpha_2)$.
\end{exr}

\begin{exr}{}
Using integration by parts, show that the gamma function $\Gamma(t) = \int_0^\infty x^{t-1}e^{-x} dx$ satisfies the relation $\Gamma (t+1) = t \Gamma (t)$ for $t > 0$.
\end{exr}

\begin{exr}{}
Prove the following results about vector expectations and variance:

\begin{enumerate}[(a)]
\item $E(Ax) = A E(x)$

\item $Cov(Ax) = AVA'$

\item $Cov(a'x, b'x) = a'Vb$

\end{enumerate}
\end{exr}

\clearpage
\newpage
\section{Further Applications to Statistics: Normal Theory and F-test}


\subsection{Bivariate Normal Distribution}
Suppose $X$ is a vector of continuous random variables and $Y = AX +
c$, where $A$ is an invertible matrix and $c$ is a constant vector. If $X$ has probability
density function $f_X$, then the probability density function of $Y$
is given by $$f_Y(y) = |det(A)|^{-1} f_X(A^{-1}(Y-c)).$$ The proof of
this result can be found in appendix B.2.1 of Bickel and Doksum.

We say that $2 \times 1$ vector $X = \left[ \begin{array} {c} X_1 \\ X_2
  \end{array} \right]$ has a \textbf{bivariate normal} distribution if
$\exists$ $Z_1, Z_2$ I.I.D $N(0, 1)$, s.t. $X = AZ + \mu$. In what
follows we will moreover assume that $A$ is invertible. You should
check at this point for yourself that $X_1 \sim N(\mu_1, \sigma_1)$
and $X_2 \sim N(\mu_2, \sigma_2)$, where $\sigma_1 = \sqrt{{a_{11}}^2 +
  {a_{12}}^2}$ and $\sigma_2 = \sqrt{{a_{21}}^2 + {a_{22}}^2}$, and
that $Cov(X_1, X_2) = a_{11}a_{21} + a_{12} + a_{22}$. We
then say that $X \sim N(\mu, \Sigma)$, where $$\Sigma = AA' =
\left[ \begin{array} {l r} \sigma_1^2 & \rho \sigma_1 \sigma_2 \\ \rho
    \sigma_1 \sigma_2 & \sigma_2^2 \end{array} \right]$$ and $\rho =
\frac{Cov(X_1, X_2)}{\sigma_1 \sigma_2}$ (you should verify that the
entries of $\Sigma = AA'$ are as we claim). The meaning behind this
definition is made explicit by the following theorem:\\

\textbf{Theorem}: Suppose $\sigma_1 \neq 0 \neq \sigma_2$ and $|\rho|<
1$. Then \\
\begin{center}
$f_X (x) = \frac{1}{2 \pi \sqrt{det(\Sigma)}}$ $\exp \left\{-
  \frac{1}{2} (x - \mu)' \Sigma^{-1} (x - \mu) \right\}$.
\end{center}

\textbf{Proof}
Note first of all that if $A$ is invertible, then it follows directly
that $\sigma_1 \neq 0 \neq \sigma_2$ and $|\rho| < 1$ (why?). Also,
$$\sqrt{det(\Sigma}) = \sqrt{det(AA')} = \sqrt{det(A) ^2} = |det(A)| =
\sigma_1 \sigma_2 \sqrt{1 - \rho^2}$$ (you should verify the last
step). We know that $f_Z(z) = \frac{1}{2 \pi} \exp \left( - \frac{1}{2}
  z'z \right)$ and since $X = AZ + \mu$ we have by the result above:
\begin{align*}
f_X(x) &= \frac{1}{2 \pi |det(A)|} \exp \left( - \frac{1}{2} (A^{-1}(x -
    \mu))'(A^{-1}(x- \mu)) \right) \\
    &= \frac{1}{2 \pi |det(A)|} \exp
    \left( - \frac{1}{2} (x - \mu)'(A^{-1})'(A^{-1})(x- \mu) \right) \\
   &=\frac{1}{2 \pi |det(A)|} \exp \left( - \frac{1}{2} (x -
    \mu)'(AA')^{-1}(x- \mu) \right) \\
 &= \frac{1}{2 \pi \sqrt{det(\Sigma})} \exp
    \left( - \frac{1}{2} (x - \mu)'\Sigma^{-1}(x- \mu) \right)
\end{align*}
which proves the theorem. The symmetric matrix $\Sigma$ is the
covariance matrix of $X$. $\Box$ \\

You should prove for yourself (Exercise 8.1) that if $X$ has a bivariate normal
distribution $N(\mu, V$, and $B$ is invertible, then $Y = BX + d$ has
a bivariate normal distribution $N(B\mu + d, BVB')$.


These results generalize to more than two variables and lead to
multivariate normal distributions. You can familiarize yourself
with some of the extensions in appendix B.6 of Bickel and Doksum. In
particular,we note here that  if $x$ is a $k \times 1$ vector of IID
$N(0, \sigma^2)$ random variables, then $Ax$ is distributed as a
multivariate $N(0, \sigma^2 AA')$ random vector.

\subsection{F-test}

We will need a couple more results about quadratic forms:
\begin{enumerate}
\item Suppose $k \times k$ $A$ is symmetric and idempotent and $k
  \times 1$ $x \sim N(0_{k\times 1}, \sigma^2 I_{k \times k})$. Then
  $\frac{x'Ax}{\sigma^2} \sim {\chi_r}^2$, where $r = rank(A)$.

\emph{Proof}: We write
  $\frac{x'Ax}{\sigma^2} = \frac{x'Q}{\sigma} \Lambda
  \frac{Q'x}{\sigma}$ and we note that $\frac{Q'x}{\sigma} \sim N(0,
  \frac{1}{\sigma^2} \times \sigma^2 Q'Q) = N(0, I)$,
  i.e.$\frac{Q'x}{\sigma}$ is a vector of IID $N(0,1)$ random
  variables. We also know that the $\Lambda$ is diagonal and its main
  diagonal consist of $r$ 1's and $k-r$ 0's, where $r = rank(A)$. You
  can then easily see from matrix multiplication  that
  $\frac{x'Q}{\sigma} \Lambda \frac{Q'x}{\sigma} = {z_1}^2 + {z_2}^2 +
  \ldots + {z_r}^2$, where the $z_i$'s are IID $N(0,1)$. Therefore
  $\frac{x'Ax}{\sigma^2} \sim {\chi_r}^2$. $\Box$

\item The above result generalizes further: suppose $k \times1$ $x
  \sim N(0, V)$, and $k \times k$ symmetric $A$ is s.t. $V$ is
  positive definite and either $AV$ or $VA$ is
  idempotent. Then $x'Ax \sim {\chi_r}^2$, where $r =
  rank(AV)$ or $rank(VA)$, respectively.

\emph{Proof}: We will prove it for the case
  of idempotent $AV$ and the proof for idempotent $VA$ is essentially
  the same. We know that $x \sim V^{1/2}z$, where $z \sim N(0, I_{k
  \times k})$, and we know that $V^{1/2} = (V^{1/2})'$, so we have:
  $x'Ax = z'(V^{1/2})' A V^{1/2} z = z' V^{1/2} A V^{1/2} z$. Consider
  $B = V^{1/2} A V^{1/2}$. $B$ is symmetric, and $B^2 = V^{1/2} A
  V^{1/2} V^{1/2} A V^{1/2} = V^{1/2} A V A V V^{-1/2} = V^{1/2} A V
  V^{-1/2} = V^{1/2} A V^{1/2} = B$, so $B$ is also idempotent. Then
  from the previous result (with $\sigma = 1$), we have $z'Bz \sim
  {\chi_r}^2$, and therefore $x'Ax \sim {\chi_r}^2$, where $r =
  rank(B) = rank (V^{1/2} A V^{1/2})$. It is a good exercise (Exercise 8.2) to
  show that $rank(B) = rank (AV)$. $\Box$

\item Let $U = x'Ax$ and $V = x'Bx$. Then the two quadratic forms are
  independent (in the probabilistic sense of the word) if $AVB =
  0$. We will not prove this result, but we will use it.
\end{enumerate}

Recall (Section 4.2)
that we had a model $Y = X\beta + \epsilon$, where $Y$ is $n \times 1$
vector of observations, $X$ is $n \times p$ matrix of explanatory
variables (with linearly independent columns), $\beta$ is $p \times 1$
vector of coefficients that we're interested in estimating, and
$\epsilon$ is $n \times 1$ vector of error terms with $E(\epsilon) =
0$. Recall that we estimate $\hat{\beta} = (X'X)^{-1}X'Y$, and we
denote fitted values $\hat{Y} = X \hat{\beta} = HY$, where the hat matrix $H=
X(X'X)^{-1}X'$ is the projection matrix onto columns of $X$, and $e =
Y - \hat{Y} = (I-H)Y$ is the vector of residuals. Recall also that
$X'e = 0$. Suppose now that $\epsilon \sim N(0, \sigma^2 I)$, i.e. the
errors are IID $N(0, \sigma^2)$ random variables. Then we can derive
some very useful distributional results:
\begin{enumerate}
\item $\hat{Y} \sim N(X \beta, \sigma^2 H)$.

\emph{Proof}: Clearly, $Y \sim
  N(X \beta, \sigma^2 I)$, and $\hat{Y} = HY$ $\Longrightarrow$
  $\hat{Y} \sim N(HX \beta, H \sigma^2 I H') = N(X(X'X)^{-1}X'X \beta,
  \sigma^2 HH') = N(X \beta, \sigma^2 H)$. $\Box$

\item $e \sim N(0, \sigma^2 (I-H))$.

\emph{Proof}: Analagous to 1.

\item $\hat{Y}$ and $e$ are independent (in probabilistic sense of the
  word).

\emph{Proof}: $Cov(\hat{Y}, e) = Cov(HY, (I-H)Y) = H (var(Y)) (I-H) =
  H \sigma^2 I (I-H) = \sigma^2 H(I-H) = 0$. And since both vectors
  were normally distributed, zero correlation implies
  independence. Notice that $Cov$ above referred to the
  cross-covariance matrix. $\Box$

\item $\frac{\|e\|^2}{\sigma^2} \sim {\chi^2}_{n-p}$.

\emph{Proof}: First
  notice that $e = (I - H)Y = (I - H)(X \beta + \epsilon) =
  (I-H)\epsilon$ (why?). Now, $\frac{\|e\|^2}{\sigma^2} =
  \frac{e'e}{\sigma^2} = \frac{\epsilon' (I-H)'(I-H)
  \epsilon}{\sigma^2} = \frac{\epsilon' (I-H)
  \epsilon}{\sigma^2}$. Since $(I-H)$ is symmetric and idempotent, and
  $\epsilon \sim N(0, \sigma^2)$, by one of the above results we have
  $\frac{\epsilon' (I-H) \epsilon}{\sigma^2} \sim {\chi_r}^2$, where
  $r = rank(I-H)$. But we know (why?) that $rank(I-H) = tr(I-H) = tr(I
  - X(X'X)^{-1}X') = tr(I) - tr(X(X'X)^{-1}X') = n - tr(X'X(X'X)^{-1})
  = n - tr(I_{p \times p}) = n - p$. So we have
  $\frac{\|e\|^2}{\sigma^2} \sim {\chi^2}_{n-p}$, and in particular
  $E(\frac{\|e\|^2}{n-p}) = \sigma^2$.  $\Box$
\end{enumerate}

Before we introduce the F-test, we are going to establish one fact
about partitioned matrices. Suppose we partition $X = [X_1\; X_2]$. Then
$[X_1\; X_2] = X(X'X)^{-1}X'[X_1\; X_2]$ $\Longrightarrow$ $X_1 =
X(X'X)^{-1}X'X_1$ and $X_2 = X(X'X)X'X_2$ (by straightforward matrix
multiplicaiton) or $HX_1 = X_1$ and $HX_2 = X_2$. Taking transposes we
also obtain $X_1 ' = X_1 ' X(X'X)^{-1}X'$ and $X_2 ' = X_2'
X(X'X)^{-1}X'$. Now suppose we want to test a theory that the last
$p_2$ coefficients of $\beta$ are actually zero (note that if we're
interested in coefficients scattered throught $\beta$, we can just
re-arrange the columns of $X$). In other words, splitting our system
into $Y = X_1 \beta_1 + X_2 \beta_2 + \epsilon$, with $n \times p_1$
$X_1$ and $n \times p_2$ $X_2$ ($p_1 + p_2 = p$), we want to see if
$\beta_2 = 0$. \\

We consider the test statistic $$\frac{\|\hat{Y}_f\|^2 -
  \|\hat{Y}_r\|^2}{\sigma^2} = \frac{Y'(X(X'X)^{-1}X' - X_1 (X_1 '
  X_1)^{-1}X_1 ')Y}{\sigma^2},$$ where $\hat{Y}_f$ is the
vector of fitted values when we regress with respect to all
columns of $X$ (full system), and $\hat{Y}_r$ is the vector of fited
  values when we regress with respect to only first $p_1$
columns of $X$ (restricted system). Under null hypothesis ($\beta_2 =
0$), we have $Y = X_1 \beta_1 + \epsilon$, and expanding the
numerator of the expression above, we get $$Y'(X(X'X)^{-1}X' - X_1 (X_1
' X_1)^{-1} X_1')Y$$ $$=\epsilon'(X(X'X)^{-1}X' - X_1 (X_1 ' X_1)^{-1}
X_1 ') \epsilon + \beta_1' X_1' (X(X'X)^{-1}X' - X_1 (X_1' X_1)^{-1}
X_1') X_1 \beta_1.$$ We recognize the second summand as $$(\beta_1' X_1'
X(X'X)^{-1}X' - \beta_1' X_1' X_1 (X_1' X_1)^{-1} X_1')X_1 \beta_1 =
(\beta_1' X_1 ' - \beta_1' X_1')X_1 \beta_1 = 0.$$ So, letting $A =
X(X'X)^{-1}X' - X_1 (X_1' X_1)^{-1} X_1'$, under null
hypothesis our test statistic is $\frac{\epsilon' A
  \epsilon}{\sigma^2}$. You should prove for yourself (Exercise 8.3) that $A$ is
symmetric and idempotent of rank $p_2$, and therefore $\frac{\epsilon' A
  \epsilon}{\sigma^2} \sim {\chi^2}_{p_2}$. That doesn't help us all that much yet since we don't know
the value of $\sigma^2$.

We have already established above that $\frac{\|e_f\|^2}{\sigma^2} \sim
{\chi^2}_{n-p}$, where $\|e_f\|^2 = \epsilon'(I-H) \epsilon$. We
proceed to show now that the two quadratic forms $\epsilon ' (I-H)
\epsilon$ and $\epsilon' A \epsilon$ are independent, by showing that
$(I-H) \sigma^2 I A = \sigma^2 (I-H)A = 0$. The proof is left as an
exercise for you. We will now denote $\frac{\|e_f\|^2}{n-p}$ by
$MS_{Res}$, and we conclude that under the null hypothesis $$\frac{\epsilon' A
  \epsilon}{p_2 \sigma^2} \left/\frac{\epsilon'(I-H) \epsilon}{(n-p)
  \sigma^2}\right.  = \frac{\|\hat{Y}_f\|^2 - \|\hat{Y}_r\|^2}{p_2 MS_{Res}}
  \sim F_{p_2, n-p}.$$ We can now test our null hypothesis $\beta_2 =
  0$, using this statistic, and we would reject for large values of $F$.

\subsection*{Exercises}
\addcontentsline{toc}{subsection}{Exercises}

\begin{exr}{}
Show that if $X$ has a bivariate normal
distribution $N(\mu, V)$, and $B$ is invertible, then $Y = BX + d$ has
a bivariate normal distribution $N(B\mu + d, BVB')$.
\end{exr}

\begin{exr}{}
  Assume $V$ is positive definite, $AV$, and $B=V^{\frac{1}{2}}AV^{\frac{1}{2}}$ is idempotent. Show that $rank(B) = rank (AV)$ (hint: consider the nullspaces, and
  invertible transformation $v = V^{1/2} w$).
\end{exr}


\begin{exr}{}
Let $X = [X_1 \, X_2]'$ for $n \times p_1$
$X_1$ and $n \times p_2$ $X_2$, and $A = X(X'X)^{-1}X' - X_1 (X_1' X_1)^{-1} X_1'$. Show that $A$ is symmetric and idempotent of rank $p_2$ (use trace to determine rank of $A$).
\end{exr}

\clearpage
\newpage
\section{References}
\begin{enumerate}
\item Bickel, Peter J and Doksum, Kjell A., 'Mathematical Statistics:
  Basic Ideas and Selected Topics', 2nd ed., 2001, Prentice Hall
\item Casella, George and Berger, Roger L, 'Statistical Inference', 2nd ed., 2001, Duxbury Press
\item Freedman, David A., 'Statistical Models: Theory and Applications',
2005, Cambridge University Press
\item Montgomery, Douglas C. et al, 'Introduction to Linear Regression
  Analysis', 3rd ed., 2001, John Wiley $\&$ Sons
\item Horn, Roger A; Johnson, Charles R., 'Matrix Analysis', 1985, Cambridge University Press
\item Strang, G, 'Linear Algebra and Its Applications', 3rd ed., 1988,
  Saunders College Publishing
\end{enumerate}







 \end{document}








